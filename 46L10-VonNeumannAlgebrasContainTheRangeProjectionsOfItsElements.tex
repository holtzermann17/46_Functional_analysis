\documentclass[12pt]{article}
\usepackage{pmmeta}
\pmcanonicalname{VonNeumannAlgebrasContainTheRangeProjectionsOfItsElements}
\pmcreated{2013-03-22 17:28:57}
\pmmodified{2013-03-22 17:28:57}
\pmowner{asteroid}{17536}
\pmmodifier{asteroid}{17536}
\pmtitle{von Neumann algebras contain the range projections of its elements}
\pmrecord{5}{39869}
\pmprivacy{1}
\pmauthor{asteroid}{17536}
\pmtype{Result}
\pmcomment{trigger rebuild}
\pmclassification{msc}{46L10}
\pmclassification{msc}{47A05}

% this is the default PlanetMath preamble.  as your knowledge
% of TeX increases, you will probably want to edit this, but
% it should be fine as is for beginners.

% almost certainly you want these
\usepackage{amssymb}
\usepackage{amsmath}
\usepackage{amsfonts}

% used for TeXing text within eps files
%\usepackage{psfrag}
% need this for including graphics (\includegraphics)
%\usepackage{graphicx}
% for neatly defining theorems and propositions
%\usepackage{amsthm}
% making logically defined graphics
%%%\usepackage{xypic}

% there are many more packages, add them here as you need them

% define commands here

\begin{document}
{\bf \PMlinkescapetext{Proposition} -} Let $T$ be an operator in a von Neumann algebra $\mathcal{M}$ acting on an Hilbert space $H$. Then the orthogonal projection onto the range of $T$ and the orthogonal projection onto the kernel of $T$ both belong to $\mathcal{M}$.

{\bf Proof :} Let $T=VR$ be the polar decomposition of $T$ with $KerV=KerR$.

By the result on the \PMlinkname{parent entry}{PolarDecompositionInVonNeumannAlgebras} we see that $V \in \mathcal{M}$.

As $V$ is a partial isometry, $VV^*$ is the (\PMlinkescapetext{orthogonal}) projection onto the range of $T$, and $I-V^*V$ is the (\PMlinkescapetext{orthogonal}) projection onto the kernel of $T$, where $I$ is the identity operator in $\mathcal{M}$.

Therefore the (\PMlinkescapetext{orthogonal}) projections onto the range and kernel of $T$ both belong to $\mathcal{M}$. $\square$
%%%%%
%%%%%
\end{document}
