\documentclass[12pt]{article}
\usepackage{pmmeta}
\pmcanonicalname{ProofOfNeumannSeriesInBanachAlgebras}
\pmcreated{2013-03-22 17:32:40}
\pmmodified{2013-03-22 17:32:40}
\pmowner{FunctorSalad}{18100}
\pmmodifier{FunctorSalad}{18100}
\pmtitle{proof of Neumann series in Banach algebras}
\pmrecord{5}{39944}
\pmprivacy{1}
\pmauthor{FunctorSalad}{18100}
\pmtype{Proof}
\pmcomment{trigger rebuild}
\pmclassification{msc}{46H05}

% this is the default PlanetMath preamble.  as your knowledge
% of TeX increases, you will probably want to edit this, but
% it should be fine as is for beginners.

% almost certainly you want these
\usepackage{amssymb}
\usepackage{amsmath}
\usepackage{amsfonts}

% used for TeXing text within eps files
%\usepackage{psfrag}
% need this for including graphics (\includegraphics)
%\usepackage{graphicx}
% for neatly defining theorems and propositions
%\usepackage{amsthm}
% making logically defined graphics
%%%\usepackage{xypic}

% there are many more packages, add them here as you need them

% define commands here

\begin{document}
Let x be an element of a Banach algebra with identity, $\|x\| < 1$. By applying the properties of the Norm in a Banach algebra, we see that the partial sums form a Cauchy sequence: $\| \sum_{n=l}^m x^n \| \leq \sum_{n=l}^m \| x\|^n  \to 0$ for $l,m \to \infty$ (as is well known from real analysis), so by completeness of the Banach Algebra, the series converges to some element $y=\sum_{n=0}^\infty x^n$.

We observe that for any $m \in \mathbb{N}$,

\begin{equation}
(1-x) \sum_{n=0}^m x^n = \sum_{n=0}^m x^n - \sum_{n=1}^{m+1} x^n = 1-x^{m+1}
\end{equation}

Furthermore, $\| x^{m+1} \| \leq \| x \|^{m+1}$, so ${\lim}_m x^{m+1} = 0$.

Thus, by taking the limit $m \to \infty$ on both sides of (1), we get

$$(1-x) y = 1$$

(We can exchange the limit with the multiplication by $(1-x)$, since the multiplication in Banach algebras is continuous)

Since the Banach algebra generated by a single element is commutative and $(1-x)$ and $y$ are both in the Banach algebra generated by $x$, we also get $y (1-x) = 1$. Hence, $y = (1-x)^{-1}$.

As in the first paragraph, the last claim $y \leq \frac{1}{1-\|y\|}$ again follows by applying the geometric series for real numbers.
%%%%%
%%%%%
\end{document}
