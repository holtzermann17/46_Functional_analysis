\documentclass[12pt]{article}
\usepackage{pmmeta}
\pmcanonicalname{LaplaceTransformOfDiracDelta}
\pmcreated{2013-03-22 19:10:56}
\pmmodified{2013-03-22 19:10:56}
\pmowner{pahio}{2872}
\pmmodifier{pahio}{2872}
\pmtitle{Laplace transform of Dirac delta}
\pmrecord{11}{42090}
\pmprivacy{1}
\pmauthor{pahio}{2872}
\pmtype{Result}
\pmcomment{trigger rebuild}
\pmclassification{msc}{46E20}
\pmclassification{msc}{44A10}
\pmclassification{msc}{34L40}

% this is the default PlanetMath preamble.  as your knowledge
% of TeX increases, you will probably want to edit this, but
% it should be fine as is for beginners.

% almost certainly you want these
\usepackage{amssymb}
\usepackage{amsmath}
\usepackage{amsfonts}

% used for TeXing text within eps files
%\usepackage{psfrag}
% need this for including graphics (\includegraphics)
%\usepackage{graphicx}
% for neatly defining theorems and propositions
 \usepackage{amsthm}
% making logically defined graphics
%%%\usepackage{xypic}

% there are many more packages, add them here as you need them

% define commands here

\theoremstyle{definition}
\newtheorem*{thmplain}{Theorem}

\begin{document}
The \PMlinkname{Dirac delta}{DiracDeltaFunction} $\delta$ can be interpreted as a linear functional, i.e. a linear mapping from a function space, consisting e.g. of certain real functions, to $\mathbb{R}$ (or $\mathbb{C}$), having the property
$$\delta[f] \;=\; f(0).$$
One may think this as the inner product
$$\langle f,\,\delta\rangle \;=\; \int_0^\infty\!f(t)\delta(t)\,dt$$
of a function $f$ and another ``function'' $\delta$, when the well-known \PMlinkescapetext{formula}
$$\int_0^\infty\!f(t)\delta(t)\,dt \;=\; f(0)$$
is true.\, Applying this to\, $f(t) := e^{-st}$,\, one gets
$$\int_0^\infty\!e^{-st}\delta(t)\,dt \;=\; e^{-0},$$
i.e. the Laplace transform
\begin{align}
\mathcal{L}\{\delta(t)\} \;=\; 1.
\end{align}
By the delay theorem, this result may be generalised to
$$\mathcal{L}\{\delta(t\!-\!a))\} \;=\; e^{-as}.$$\\



When introducing some ``nascent Dirac delta function'', for example
\begin{align*}
\eta_\varepsilon(t) \;:=\; 
\begin{cases}
\frac{1}{\varepsilon} \quad \mbox{for}\;\; 0 \le t \le \varepsilon,\\
0 \quad \mbox{for} \qquad t > \varepsilon,
\end{cases}
\end{align*}
as an ``approximation'' of Dirac delta, we obtain the Laplace transform
$$\mathcal{L}\{\eta_\varepsilon(t)\} \;=\; \int_0^\infty\!e^{-st}\eta_\varepsilon(t)\,dt
\;=\; \int_0^\varepsilon\frac{e^{-st}}{\varepsilon}\,dt+\int_\varepsilon^\infty\!e^{-st}\cdot0\,dt 
\;=\; \frac{1}{\varepsilon}\int_0^\varepsilon\!e^{-st}\,dt \;=\; \frac{1\!-\!e^{-\varepsilon s}}{\varepsilon s}.$$
As the Taylor expansion shows, we then have
$$\lim_{\varepsilon\to0+}\mathcal{L}\{\eta_\varepsilon(t)\} \;=\; 1,$$
being in accordance with (1).

%%%%%
%%%%%
\end{document}
