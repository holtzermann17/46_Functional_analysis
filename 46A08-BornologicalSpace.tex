\documentclass[12pt]{article}
\usepackage{pmmeta}
\pmcanonicalname{BornologicalSpace}
\pmcreated{2013-03-22 15:59:09}
\pmmodified{2013-03-22 15:59:09}
\pmowner{Mathprof}{13753}
\pmmodifier{Mathprof}{13753}
\pmtitle{bornological space}
\pmrecord{8}{38003}
\pmprivacy{1}
\pmauthor{Mathprof}{13753}
\pmtype{Definition}
\pmcomment{trigger rebuild}
\pmclassification{msc}{46A08}
\pmdefines{bornivore}

% this is the default PlanetMath preamble.  as your knowledge
% of TeX increases, you will probably want to edit this, but
% it should be fine as is for beginners.

% almost certainly you want these
\usepackage{amssymb}
\usepackage{amsmath}
\usepackage{amsfonts}

% used for TeXing text within eps files
%\usepackage{psfrag}
% need this for including graphics (\includegraphics)
%\usepackage{graphicx}
% for neatly defining theorems and propositions
%\usepackage{amsthm}
% making logically defined graphics
%%%\usepackage{xypic}

% there are many more packages, add them here as you need them

% define commands here

\begin{document}
A {\it bornivore} is a set which absorbs all bounded sets.
That is, $G$ is a bornivore if given any bounded set $B$, there exists a $\delta > 0$ such that
$\epsilon B \subset G$ for $0 \leq \epsilon < \delta$.

A locally convex topological vector space is said to be {\it bornological} if every convex bornivore is a neighborhood of 0.

A metrizable topological vector space is bornological.

\begin{thebibliography}{99}
\bibitem{wil} A. Wilansky, \emph{Functional Analysis}, Blaisdell Publishing Co. 1964. 
\bibitem{sch} H.H. Schaefer, M. P. Wolff, \emph{Topological Vector Spaces},
2nd ed. 1999, Springer-Verlag.
\end{thebibliography}

%%%%%
%%%%%
\end{document}
