\documentclass[12pt]{article}
\usepackage{pmmeta}
\pmcanonicalname{WienerAlgebra}
\pmcreated{2013-03-22 17:22:55}
\pmmodified{2013-03-22 17:22:55}
\pmowner{asteroid}{17536}
\pmmodifier{asteroid}{17536}
\pmtitle{Wiener algebra}
\pmrecord{10}{39748}
\pmprivacy{1}
\pmauthor{asteroid}{17536}
\pmtype{Definition}
\pmcomment{trigger rebuild}
\pmclassification{msc}{46J10}
\pmclassification{msc}{43A50}
\pmclassification{msc}{42A20}
\pmclassification{msc}{46K05}
\pmdefines{Wiener theorem}

\endmetadata

% this is the default PlanetMath preamble.  as your knowledge
% of TeX increases, you will probably want to edit this, but
% it should be fine as is for beginners.

% almost certainly you want these
\usepackage{amssymb}
\usepackage{amsmath}
\usepackage{amsfonts}

% used for TeXing text within eps files
%\usepackage{psfrag}
% need this for including graphics (\includegraphics)
%\usepackage{graphicx}
% for neatly defining theorems and propositions
%\usepackage{amsthm}
% making logically defined graphics
%%%\usepackage{xypic}

% there are many more packages, add them here as you need them

% define commands here

\begin{document}
\subsubsection{Definition and classification of the Wiener algebra}
Let $W$ be the space of all complex functions on $[0,2\pi[$ whose Fourier series converges
 absolutely, that is, all functions $f:[0,2\pi[ \longrightarrow \mathbb{C}$ whose Fourier series
\begin{displaymath}
f(t)= \sum_{n=-\infty}^{+\infty} \hat{f}(n)e^{int}
\end{displaymath}
is such that $\sum_{n} |\hat{f}(n)| < \infty$ .

Under pointwise operations and the norm $\|f \| = \sum_{n} |\hat{f}(n)|\,,$ $W$ is a commutative Banach
 algebra of continuous functions, with an identity element. $W$ is usually called the {\bf Wiener algebra}.

{\bf Theorem -} $W$ is isometrically isomorphic to the Banach algebra $\ell^1(\mathbb{Z})$ with the
 convolution product. The isomorphism is given by:
\begin{displaymath}
(a_k)_{k \in \mathbb{Z}} \longleftrightarrow f(t) = \sum_{n=-\infty}^{+\infty} a_k e^{int}
\end{displaymath}

\subsubsection{Wiener's Theorem}

{\bf Theorem (Wiener) -} If $f \in W$ has no zeros then $1/f \in W$, that is, $1/f$ has an
 absolutely convergent Fourier series.

{\bf Proof :} We want to prove that $f$ is invertible in $W$. As $W$ is commutative, that is the same as proving
 that $f$ does not belong to any maximal ideal of $W$. Therefore we only need to show that $f$ is not in the
 kernel of any multiplicative linear functional of $W$.

Let $\phi$ be a multiplicative linear functional in $W$. We have that

\begin{displaymath}
\phi (f)= \phi \Big( \sum_{n=-\infty}^{+\infty} \hat{f}(n)e^{int} \Big) = \sum_{n=-\infty}^{+\infty} \hat{f}(n)\phi (e^{int}) = \sum_{n=-\infty}^{+\infty} \hat{f}(n)\phi^n (e^{it})
\end{displaymath}

Since $\|\phi\|=1$ we have that
\begin{displaymath}
|\phi (e^{it})| \le \|\phi\|\|e^{it}\|=\|e^{it}\|=1
\end{displaymath}
and
\begin{displaymath}
|\phi (e^{-it})| \le \|\phi\|\|e^{-it}\|=\|e^{-it}\|=1
\end{displaymath}

Since $1=|\phi(e^{it}e^{-it})|=|\phi(e^{it})||\phi(e^{-it})|$ we deduce that
\begin{displaymath}
|\phi (e^{it})|=1
\end{displaymath}

We can conclude that
\begin{center}
$\phi(e^{it})=e^{it_0}\;$ for some $t_0 \in [0,2\pi[$
\end{center}

Therefore we obtain
\begin{displaymath}
\phi (f)= \sum_{n=-\infty}^{+\infty} \hat{f}(n)e^{int_0} = f(t_0)
\end{displaymath}
which is non-zero by definition of $f$.

We conclude that $f$ does not belong to the kernel of any multiplicative linear functional $\phi$. $\square$

\subsubsection{Remark}
The Wiener algebra is a Banach *-algebra with the involution given by $f^*(t):=\overline{f(-t)}$, but it is not a \PMlinkname{$C^*$-algebra}{CAlgebra} under this involution.
%%%%%
%%%%%
\end{document}
