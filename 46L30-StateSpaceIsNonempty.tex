\documentclass[12pt]{article}
\usepackage{pmmeta}
\pmcanonicalname{StateSpaceIsNonempty}
\pmcreated{2013-03-22 17:45:14}
\pmmodified{2013-03-22 17:45:14}
\pmowner{asteroid}{17536}
\pmmodifier{asteroid}{17536}
\pmtitle{state space is non-empty}
\pmrecord{14}{40205}
\pmprivacy{1}
\pmauthor{asteroid}{17536}
\pmtype{Theorem}
\pmcomment{trigger rebuild}
\pmclassification{msc}{46L30}
\pmclassification{msc}{46L05}

% this is the default PlanetMath preamble.  as your knowledge
% of TeX increases, you will probably want to edit this, but
% it should be fine as is for beginners.

% almost certainly you want these
\usepackage{amssymb}
\usepackage{amsmath}
\usepackage{amsfonts}

% used for TeXing text within eps files
%\usepackage{psfrag}
% need this for including graphics (\includegraphics)
%\usepackage{graphicx}
% for neatly defining theorems and propositions
%\usepackage{amsthm}
% making logically defined graphics
%%%\usepackage{xypic}

% there are many more packages, add them here as you need them

% define commands here

\begin{document}
\PMlinkescapeword{self-adjoint}
\PMlinkescapeword{restriction}

In this entry we prove the existence of states for every \PMlinkname{$C^*$-algebra}{CALgebra}.

{\bf Theorem -} Let $\mathcal{A}$ be a $C^*$-algebra. For every \PMlinkname{self-adjoint}{InvolutaryRing} element $a \in \mathcal{A}$ there exists a state $\psi$ on $\mathcal{A}$ such that $|\psi(a)|=\|a\|$.

$\;$

{\bf \emph{Proof :}} We first consider the case where $\mathcal{A}$ is \PMlinkname{unital}{Ring}, with identity element $e$.

Let $\mathcal{B}$ be the $C^*$-subalgebra generated by $a$ and $e$. Since $a$ is self-adjoint, $\mathcal{B}$ is a comutative $C^*$-algebra with identity element.

Thus, by the Gelfand-Naimark theorem, $\mathcal{B}$ is isomorphic to $C(X)$, the space of continuous functions $X \longrightarrow \mathbb{C}$ for some compact set $X$.

Regarding $a$ as an element of $C(X)$, $a$ attains a maximum at a point $x_0 \in X$, since $X$ is compact. Hence, $\|a\| = |a(x_0)|$.

The evaluation function at $x_0$,
\begin{align*}
ev_{x_0}: C(X) \longrightarrow \mathbb{C}\\
ev_{x_0}(f):= f(x_0)
\end{align*}
is a multiplicative linear functional of $C(X)$. Hence, $\|ev_{x_0}\|=1$ and also $|ev_{x_0}(a)|= |a(x_0)|=\|a\|$.

We can now extend $ev_{x_0}$ to a linear functional $\psi$ on $\mathcal{A}$ such that $\|\psi\| = \|ev_{x_0}\|=1$, using the Hahn-Banach theorem.

Also, $\psi(e) = ev_{x_0}(e)=1$ and so $\psi$ is a norm one positive linear functional, i.e. $\psi$ is a state on $\mathcal{A}$.

Of course, $\psi$ is such that $|\psi(a)| = |ev_{x_0}(a)| = \|a\|$.

In case $\mathcal{A}$ does not have an identity element we can consider its minimal unitization $\widetilde{\mathcal{A}}$. By the preceding \PMlinkescapetext{argument} there is a state $\widetilde{\psi}$ on $\widetilde{\mathcal{A}}$ satisfying the required \PMlinkescapetext{property}. Now, we just need to take the \PMlinkname{restriction}{RestrictionOfAFunction} of $\widetilde{\psi}$ to $\mathcal{A}$ and this restriction is a state in $\mathcal{A}$ satisfying the required \PMlinkescapetext{property}. $\square$
%%%%%
%%%%%
\end{document}
