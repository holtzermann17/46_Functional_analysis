\documentclass[12pt]{article}
\usepackage{pmmeta}
\pmcanonicalname{State}
\pmcreated{2013-03-22 13:50:18}
\pmmodified{2013-03-22 13:50:18}
\pmowner{mhale}{572}
\pmmodifier{mhale}{572}
\pmtitle{state}
\pmrecord{8}{34574}
\pmprivacy{1}
\pmauthor{mhale}{572}
\pmtype{Definition}
\pmcomment{trigger rebuild}
\pmclassification{msc}{46L05}
\pmrelated{ExtensionAndRestrictionOfStates}
\pmrelated{AlgebraicQuantumFieldTheoriesAQFT}
\pmdefines{pure state}
\pmdefines{tracial state}

\endmetadata

\usepackage{amssymb}
\usepackage{amsmath}
\usepackage{amsfonts}
\usepackage{amsthm}

% used for TeXing text within eps files
%\usepackage{psfrag}
% need this for including graphics (\includegraphics)
%\usepackage{graphicx}
% making logically defined graphics
%%%\usepackage{xypic}

% my maths package

\newcommand*{\Nset}{\mathbb{N}}
\newcommand*{\Zset}{\mathbb{Z}}
\newcommand*{\Qset}{\mathbb{Q}}
\newcommand*{\Rset}{\mathbb{R}}
\newcommand*{\Cset}{\mathbb{C}}
\newcommand*{\Hset}{\mathbb{H}}
\newcommand*{\Oset}{\mathbb{O}}
\newcommand*{\Bset}{\mathbb{B}}
\newcommand*{\Kset}{\mathbb{K}}
\newcommand*{\Sset}{\mathbb{S}}
\newcommand*{\Tset}{\mathbb{T}}
\newcommand*{\GLgrp}{\mathrm{GL}}
\newcommand*{\SLgrp}{\mathrm{SL}}
\newcommand*{\Ogrp}{\mathrm{O}}
\newcommand*{\SOgrp}{\mathrm{SO}}
\newcommand*{\Ugrp}{\mathrm{U}}
\newcommand*{\SUgrp}{\mathrm{SU}}
\newcommand*{\e}{\mathop{\mathrm{e}}\nolimits}
\newcommand*{\im}{\mathord{\mathrm{i}}}
\newcommand*{\identity}{\mathord{\mathrm{1\!\!\!\:I}}}
\newcommand*{\tr}{\mathop{\mathrm{tr}}}
\newcommand*{\Tr}{\mathop{\mathrm{Tr}}}
\newcommand*{\norm}[1]{\Vert #1\Vert}
\renewcommand*{\d}{\mathrm{d}}
\newcommand*{\deriv}[2]{\frac{\d #1}{\d #2}}
\newcommand*{\pderiv}[2]{\frac{\partial #1}{\partial #2}}
\newcommand*{\fderiv}[2]{\frac{\delta #1}{\delta #2}}

% my noncommutative geometry package

\newcommand*{\algebra}[1][A]{\mathord{\mathcal{#1}}}
\newcommand*{\hilbert}[1][H]{\mathord{\mathcal{#1}}}
\newcommand*{\hilbmod}[1][E]{\mathord{\mathcal{#1}}}
\newcommand*{\Matrix}[2]{\mathord{\mathrm{M}_{#1}(#2)}}
\newcommand*{\dixmier}{\mathop{\mathrm{Tr}_\omega}}
\newcommand*{\Res}{\mathop{\mathrm{Res}}}
\newcommand*{\Wres}{\mathop{\mathrm{Wres}}}
\newcommand*{\Aut}{\mathop{\mathrm{Aut}}\nolimits}
\newcommand*{\Inn}{\mathop{\mathrm{Inn}}\nolimits}
\newcommand*{\Out}{\mathop{\mathrm{Out}}\nolimits}
\newcommand*{\Diff}{\mathop{\mathrm{Diff}}\nolimits}
\newcommand*{\Ker}{\mathop{\mathrm{Ker}}\nolimits}
\newcommand*{\Coker}{\mathop{\mathrm{Coker}}\nolimits}
\newcommand*{\Img}{\mathop{\mathrm{Im}}\nolimits}
\newcommand*{\End}{\mathop{\mathrm{End}}\nolimits}
\newcommand*{\spin}{\mathop{\mathrm{spin}}\nolimits}
\newcommand*{\Ind}{\mathop{\mathrm{Ind}}\nolimits}
\newcommand*{\KK}{\mathit{KK}}
\newcommand*{\HH}{\mathit{HH}}
\newcommand*{\HC}{\mathit{HC}}
\newcommand*{\ch}{\mathop{\mathrm{ch}}\nolimits}

% my category theory package

\newcommand*{\mathcat}[1]{\mathord{\mathbf{#1}}}
\newcommand*{\id}{\mathrm{id}}
\newcommand*{\op}{\mathrm{op}}
\newcommand*{\boxprod}{\mathbin{\square}}

% my environments

\newtheoremstyle{inlinedefn}{}{0pt}{}{}{\bfseries}{.}{0.5em}{}
\theoremstyle{inlinedefn}
\newtheorem{definition}{Definition}

\newtheoremstyle{break}{\baselineskip}{\baselineskip}{\itshape}{}{\bfseries}{}{\newline}{}
\theoremstyle{break}
\newtheorem{example}{Example}

% misc commands

\newcommand*{\defn}[1]{\textbf{#1}}
\begin{document}
A \textbf{state} $\Psi$ on a $C^*$-algebra $A$ is a positive linear functional
$\Psi\colon A \to \Cset$, $\Psi(a^*a) \geq 0$ for all $a \in A$, with unit norm.
The norm of a positive linear functional is defined by
\begin{equation}
\norm{\Psi} = \sup_{a \in A}\{|\Psi(a)| : \norm{a}\leq 1\}.
\end{equation}
For a unital $C^*$-algebra, $\norm{\Psi} = \Psi(\identity)$.

The space of states is a convex set.
Let $\Psi_1$ and $\Psi_2$ be states, then the convex combination
\begin{equation}
\lambda\Psi_1+(1-\lambda)\Psi_2, \quad \lambda \in [0,1],
\end{equation}
is also a state.

A state is \textbf{pure} if it is not a convex combination of two other states.
Pure states are the extreme points of the convex set of states.
A pure state on a commutative $C^*$-algebra is equivalent to a character.

A state is called a \textbf{tracial state} if it is also a trace.

When a $C^*$-algebra is represented on a Hilbert space $\hilbert$,
every unit vector $\psi \in \hilbert$ determines a (not necessarily pure) state in the form of an \defn{expectation value},
\begin{equation}
\Psi(a) = \langle\psi, a\psi\rangle.
\end{equation}
In physics, it is common to refer to such states by their vector $\psi$ rather than the linear functional $\Psi$.
The converse is not always true; not every state need be given by
an expectation value.
For example, delta functions (which are distributions not functions)
give pure states on $C_0(X)$,
but they do not correspond to any vector in a Hilbert space
(such a vector would not be square-integrable).

\begin{thebibliography}{10}
\bibitem{Murphy}
G.~Murphy, {\em $C^*$-Algebras and Operator Theory}.
\newblock Academic Press, 1990.
\end{thebibliography}
%%%%%
%%%%%
\end{document}
