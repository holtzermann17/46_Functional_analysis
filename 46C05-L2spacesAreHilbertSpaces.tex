\documentclass[12pt]{article}
\usepackage{pmmeta}
\pmcanonicalname{L2spacesAreHilbertSpaces}
\pmcreated{2013-03-22 17:32:25}
\pmmodified{2013-03-22 17:32:25}
\pmowner{asteroid}{17536}
\pmmodifier{asteroid}{17536}
\pmtitle{$L^2$-spaces are Hilbert spaces}
\pmrecord{23}{39939}
\pmprivacy{1}
\pmauthor{asteroid}{17536}
\pmtype{Theorem}
\pmcomment{trigger rebuild}
\pmclassification{msc}{46C05}
\pmsynonym{square integrable functions form an Hilbert space}{L2spacesAreHilbertSpaces}
%\pmkeywords{Hilbert spaces}
%\pmkeywords{Lp space}
%\pmkeywords{Banach spaces}
%\pmkeywords{sequilinearity}
%\pmkeywords{linearity of the Lebesgue integral}
%\pmkeywords{conjugate symmetry}
%\pmkeywords{$L^p$-norm}
\pmrelated{LpSpace}
\pmrelated{HilbertSpace}
\pmrelated{MeasureSpace}
\pmrelated{BanachSpace}
\pmrelated{RieszFischerTheorem}
\pmdefines{linear space of square integrable functions}
\pmdefines{sequilinearity}

% this is the default PlanetMath preamble.  as your knowledge
% of TeX increases, you will probably want to edit this, but
% it should be fine as is for beginners.

% almost certainly you want these
\usepackage{amssymb}
\usepackage{amsmath}
\usepackage{amsfonts}

% used for TeXing text within eps files
%\usepackage{psfrag}
% need this for including graphics (\includegraphics)
%\usepackage{graphicx}
% for neatly defining theorems and propositions
%\usepackage{amsthm}
% making logically defined graphics
%%%\usepackage{xypic}

% there are many more packages, add them here as you need them

% define commands here

\begin{document}
Let $(X, \mathfrak{B}, \mu)$ be a measure space. Let $L^2(X)$ denote the \PMlinkname{$L^2$-space}{LpSpace} associated with this measure space, i.e. $L^2(X)$ consists of measurable functions $f:X \longrightarrow \mathbb{C}$ such that
\begin{displaymath}
\|f\|_2 := \left (\int_X |f|^2 d\mu \right)^{\frac{1}{2}} < \infty
\end{displaymath}
identified up to equivalence almost everywhere.

It is known that all \PMlinkname{$L^p$-spaces}{LpSpace}, with $1\leq p \leq \infty$, are Banach spaces with respect to the \PMlinkname{$L^p$-norm}{LpSpace} $\;\|\cdot\|_p$. For $L^2(X)$ we can say \PMlinkescapetext{even} more:

{\bf Theorem -} $L^2(X)$ is an Hilbert Space with respect to the inner product $\langle \cdot, \cdot \rangle$ defined by
\begin{displaymath}
\langle f, g \rangle = \int_X f\overline{g} \;d\mu
\end{displaymath}

\emph{Proof:}

 \emph{Sesquilinearity} follows from the \PMlinkname{linearity of the Lebesgue integral}{PropertiesOfTheLebesgueIntegralOfLebesgueIntegrableFunctions} (that is, the inner product defined above is linear in the first argument and conjugate linear in the second one). The conjugate symmetry is evident.

Positive definiteness holds by construction: If $\int_X |f|^2 d\mu = 0$, then $|f|^2$ (and therefore $f$) is zero almost everywhere, thus the equivalence class of $f$ is the equivalence class of the zero function (which is the additive neutral element of the space).

Completeness is proved for the general case of $L^p$-spaces in \PMlinkname{this article}{ProofThatLpSpacesAreComplete}.$\square$

\subsubsection{Remarks}
\begin{itemize}
\item The spaces $\mathbb{C}^n$ or $\mathbb{R}^n$ with the usual inner product are particular examples of $L^2(X)$, choosing $X = \{1, \dots, n\}$ with the counting measure.
\item Choosing appropriate spaces $X$ it can be shown that all Hilbert spaces are isometrically isomorphic to a $L^2$-space.
\end{itemize}
%%%%%
%%%%%
\end{document}
