\documentclass[12pt]{article}
\usepackage{pmmeta}
\pmcanonicalname{ModularMappingsInVectorSpacesOverTheFieldOfComplexNumbers}
\pmcreated{2013-03-22 16:08:12}
\pmmodified{2013-03-22 16:08:12}
\pmowner{gilbert_51126}{14238}
\pmmodifier{gilbert_51126}{14238}
\pmtitle{modular mappings in vector spaces over the field of complex numbers}
\pmrecord{11}{38208}
\pmprivacy{1}
\pmauthor{gilbert_51126}{14238}
\pmtype{Definition}
\pmcomment{trigger rebuild}
\pmclassification{msc}{46-00}
\pmdefines{modular}

\endmetadata

% this is the default PlanetMath preamble.  as your knowledge
% of TeX increases, you will probably want to edit this, but
% it should be fine as is for beginners.

% almost certainly you want these
\usepackage{amssymb}
\usepackage{amsmath}
\usepackage{amsfonts}

% used for TeXing text within eps files
%\usepackage{psfrag}
% need this for including graphics (\includegraphics)
%\usepackage{graphicx}
% for neatly defining theorems and propositions
%\usepackage{amsthm}
% making logically defined graphics
%%%\usepackage{xypic}

% there are many more packages, add them here as you need them

% define commands here

\begin{document}
\PMlinkescapeword{limit}
\PMlinkescapeword{normal}
Suppose $X$ is a $\mathbb{C}$-vector space. A mapping $\rho:X\to [0,\infty]$ is called \textit{modular} if the following three conditions are satisfied:
\begin{enumerate}
\item $\rho(x) = 0$ if and only if $x=0$.
\item $\rho(\alpha x) = \rho(x)$ for all $x \in X$ and for all scalars $\alpha$ such that $|\alpha|=1$.
\item $\rho(\alpha x + \beta y) \leq \rho(x) + \rho(y)$ for all $x,y \in X$ and for all scalars $\alpha$ and $\beta$ such that $\alpha + \beta =1$ and $\alpha ,\beta \geq 0$.
\end{enumerate}


%%%%%
%%%%%
\end{document}
