\documentclass[12pt]{article}
\usepackage{pmmeta}
\pmcanonicalname{ExampleOfNonseparableHilbertSpace}
\pmcreated{2013-03-22 15:44:25}
\pmmodified{2013-03-22 15:44:25}
\pmowner{rspuzio}{6075}
\pmmodifier{rspuzio}{6075}
\pmtitle{example of non-separable Hilbert space}
\pmrecord{6}{37691}
\pmprivacy{1}
\pmauthor{rspuzio}{6075}
\pmtype{Example}
\pmcomment{trigger rebuild}
\pmclassification{msc}{46C05}
\pmrelated{AlmostPeriodicFunction}

% this is the default PlanetMath preamble.  as your knowledge
% of TeX increases, you will probably want to edit this, but
% it should be fine as is for beginners.

% almost certainly you want these
\usepackage{amssymb}
\usepackage{amsmath}
\usepackage{amsfonts}

% used for TeXing text within eps files
%\usepackage{psfrag}
% need this for including graphics (\includegraphics)
%\usepackage{graphicx}
% for neatly defining theorems and propositions
%\usepackage{amsthm}
% making logically defined graphics
%%%\usepackage{xypic}

% there are many more packages, add them here as you need them

% define commands here
\begin{document}
As an exaple of a Hilbert space which is not separable, one may
consider the following function space:

Consider real-valued functions on the real line but, instead of the
usual $L^2$ norm, use the following inner product:
 \[ (f,g) = \lim_{R \to \infty} {1 \over R} \int_{-R}^{+R} f(x) g(x)
 \, dx \]
The first thing to note about this is that non-trivial functions have
norm 0.  For instance, any function of $L^2$ has zero norm according
to this inner product.

Define the Hilbert space as the set of equivalence classes of
functions for which this norm is finite modulo functions for which it
is zero.  Note that $\sin ax$ and $\sin bx$ are orthogonal under this
norm if $a \neq b$.  Hence, the set of functions $\sin ax$, where $a$
is a real number, form an orthonormal set.  Since the number of real
numbers is uncountable, we have an uncountably infinite orthonormal set, 
so this Hilbert space is not separable.

It is important not to confuse what we are doing here with the Fourier
integral.  In that case, we are dealing with $L^2$, the functions
$\sin ax$ have infinite $L^2$ norm (so they are not elements of that
Hilbert space) and the expansion of a function in terms of them is a
direct integral.  By contrast, in the case propounded here, the
expansion of a function of this space in terms of them would take the
form of a direct sum, just as with the Fourier series of a function on
a finite interval.
%%%%%
%%%%%
\end{document}
