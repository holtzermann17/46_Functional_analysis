\documentclass[12pt]{article}
\usepackage{pmmeta}
\pmcanonicalname{ModularSpace}
\pmcreated{2013-03-22 16:15:53}
\pmmodified{2013-03-22 16:15:53}
\pmowner{gilbert_51126}{14238}
\pmmodifier{gilbert_51126}{14238}
\pmtitle{modular space}
\pmrecord{6}{38374}
\pmprivacy{1}
\pmauthor{gilbert_51126}{14238}
\pmtype{Definition}
\pmcomment{trigger rebuild}
\pmclassification{msc}{46-00}

% this is the default PlanetMath preamble.  as your knowledge
% of TeX increases, you will probably want to edit this, but
% it should be fine as is for beginners.

% almost certainly you want these
\usepackage{amssymb}
\usepackage{amsmath}
\usepackage{amsfonts}

% used for TeXing text within eps files
%\usepackage{psfrag}
% need this for including graphics (\includegraphics)
%\usepackage{graphicx}
% for neatly defining theorems and propositions
%\usepackage{amsthm}
% making logically defined graphics
%%%\usepackage{xypic}

% there are many more packages, add them here as you need them

% define commands here

\begin{document}
\PMlinkescapeword{limit}
\PMlinkescapeword{normal}
Let $\rho$ be a modular on a real or complex vector space $X$. Then the subspace
\[
X_\rho := \{x \in X : \lim_{\lambda\rightarrow 0} \rho(\lambda x) = 0\}
\]
of $X$ is called the \emph{modular space} corresponding to the modular $\rho$.

%%\emph{Remark.}
%%Let us show that $X_\rho$ is indeed a subspace of $X$.
%%From the definition of a modular, it is clear that $0 \in X_\rho$.
%%Suppose $x \in X_\rho$ and $\alpha$ is a nonzero scalar. Then, since $|\alpha %%/|\alpha|| = 1$, we have
%%\begin{eqnarray*}
%%\lim_{\gamma\rightarrow 0} \rho(\frac_{\gamma}^{|\alpha|} \alpha x) 
%%&=& \lim_{\gamma\rightarrow 0} \rho(\frac_{\alpha}^{|\alpha|} \gamma x) \\
%%&=& \lim_{\gamma\rightarrow 0} \rho(\gamma x) \\
%%&=& 0.
%%\end{eqnarray*} 
%%Let $\lambda = \frac_{\gamma}^{|\alpha|}$, and observe that %%$\lambda\rightarrow 0$ 
%%as $\gamma\rightarrow 0$. From above, it follows that $\alpha \in X_\rho$.

%%Now suppose $x,y \in X_\rho$. Then
%%\begin{eqnarray*}
%%\lim_{\gamma\rightarrow 0} \rho(\frac_{\gamma}^{2}(x+y)) 
%%&=& \lim_{\gamma\rightarrow 0} \rho(\frac_{\gamma}^{2} x + \frac_{\gamma}^{2} %%y) \\
%%&\leq& \lim_{\gamma\rightarrow 0} \rho(\gamma x) 
%%+\lim_{\gamma\rightarrow 0} \rho(\gamma y) \\
%%&=& 0.
%%\end{eqnarray*} 
%%Hence, 
%%\[
%%\lim_{\gamma\rightarrow 0} \rho(\frac_{\gamma}^{2}(x+y)) = 0.
%%\]
%%Now let $\lambda = \gamma / 2$, and note that $\lambda\rightarrow 0$ 
%%as $\gamma\rightarrow 0$. Therefore, it follows that $x+y \in X_\rho$.
%%Indeed, $X_\rho$ is a subspace of $X$.
%%%%%
%%%%%
\end{document}
