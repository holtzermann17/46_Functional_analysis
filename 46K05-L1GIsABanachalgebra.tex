\documentclass[12pt]{article}
\usepackage{pmmeta}
\pmcanonicalname{L1GIsABanachalgebra}
\pmcreated{2013-03-22 17:42:14}
\pmmodified{2013-03-22 17:42:14}
\pmowner{asteroid}{17536}
\pmmodifier{asteroid}{17536}
\pmtitle{$L^1(G)$ is a Banach *-algebra}
\pmrecord{15}{40146}
\pmprivacy{1}
\pmauthor{asteroid}{17536}
\pmtype{Example}
\pmcomment{trigger rebuild}
\pmclassification{msc}{46K05}
\pmclassification{msc}{46H05}
\pmclassification{msc}{44A35}
\pmclassification{msc}{43A20}
\pmclassification{msc}{22D05}
\pmclassification{msc}{22A10}
\pmrelated{DualGroupOfGIsHomeomorphicToTheCharacterSpaceOfL1G}
\pmrelated{ConvolutionsOfComplexFunctionsOnLocallyCompactGroups}
\pmdefines{$L^1(\mathbb{R})$ is a Banach *-algebra}
\pmdefines{group algebra}

\endmetadata

% this is the default PlanetMath preamble.  as your knowledge
% of TeX increases, you will probably want to edit this, but
% it should be fine as is for beginners.

% almost certainly you want these
\usepackage{amssymb}
\usepackage{amsmath}
\usepackage{amsfonts}

% used for TeXing text within eps files
%\usepackage{psfrag}
% need this for including graphics (\includegraphics)
%\usepackage{graphicx}
% for neatly defining theorems and propositions
%\usepackage{amsthm}
% making logically defined graphics
%%%\usepackage{xypic}

% there are many more packages, add them here as you need them

% define commands here

\begin{document}
\PMlinkescapeword{involution}

\subsection{The Banach *-algebra $L^1(\mathbb{R})$.}

Consider the Banach space \PMlinkname{$L^1(\mathbb{R})$}{LpSpace}, i.e. the space of Borel measurable functions $f:\mathbb{R} \longrightarrow \mathbb{C}$ such that
\begin{displaymath}
\|f\|_1 := \int_{\mathbb{R}} |f(x)|\; dx < \infty
\end{displaymath}
identified up to equivalence almost everywhere.

The convolution product of functions $f, g \in L^1(\mathbb{R})$, given by
\begin{displaymath}
(f * g)(z) = \int_{\mathbb{R}} f(x)g(z-x)dx ,
\end{displaymath}
is a well-defined product in $L^1(\mathbb{R})$, i.e. $f*g \in L^1(\mathbb{R})$, that satisfies the inequality
\begin{displaymath}
\|f*g\|_1 \leq \|f\|_1\|g\|_1\;.
\end{displaymath}
Therefore, with the convolution product, $L^1(\mathbb{R})$ is a Banach algebra.

Moreover, we can define an \PMlinkname{involution}{InvolutaryRing} in $L^1(\mathbb{R})$ by $f^*(x)=\overline{f(-x)}$. With this involution $L^1(\mathbb{R})$ is Banach *-algebra.

\subsection{Generalization to $L^1(G)$.}
Let $G$ be a locally compact topological group and $\mu$ its left Haar measure. Consider the space \PMlinkname{$L^1(G)$}{LpSpace} consisting of measurable functions $f:G \longrightarrow \mathbb{C}$ such that
\begin{displaymath}
\|f\|_1 := \int_G |f|\; d\mu < \infty
\end{displaymath}
identified up to equivalence almost everywhere.

The convolution product of functions $f, g \in L^1(G)$, given by
\begin{displaymath}
(f * g)(s) = \int_G f(t)g(t^{-1}s)\;d\mu(t) ,
\end{displaymath}
is a well-defined product in $L^1(G)$, i.e. $f*g \in L^1(G)$, that satisfies the inequality
\begin{displaymath}
\|f*g\|_1 \leq \|f\|_1\|g\|_1\;.
\end{displaymath}
Therefore, with this convolution product, $L^1(G)$ is a Banach algebra.

An involution can also be defined in $L^1(G)$ by $f^*(s) = \Delta_G(s^{-1})\overline{f(s^{-1})}$, where $\Delta_G$ is the modular function of $G$.

With this product and involution $L^1(G)$ is a Banach *-algebra.

\subsection{Commutative case: the group algebra.}

The algebras $L^1(G)$ are commutative if and only if the group $G$ is commutative.

Commutative groups are of course \PMlinkname{unimodular}{UnimodularGroup2}, hence $\Delta_G (s) = 1$ for all $s \in G$.

So in the commutative case the convolution product and involution are given, respectively, by
\begin{eqnarray*}
(f * g)(s) & = & \int_G f(t)g(s-t)\;d\mu(t)\\
f^*(s) & = & \overline{f(-s)}
\end{eqnarray*}
and $L^1(G)$ is called the \emph{group algebra} of $G$.

For finite groups, the group algebra defined as above coincides with the \PMlinkname{group algebra $\mathbb{C}(G)$}{GroupRing}.

\subsection{An equivalent construction}
In the construction of $L^1(G)$ presented above we are considering equivalence classes of measurable functions on $G$ with respect to the Haar measure. To avoid this kind of measure theoretic considerations it is sometimes better to work with another (\PMlinkescapetext{equivalent}) definition of $L^1(G)$:

Let $C_c(G)$ be the space of continuous functions $G \longrightarrow \mathbb{C}$ with compact support. We can endow this space with a convolution product, an involution and a norm by setting
\begin{eqnarray*}
(f * g)(s) & = & \int_G f(t)g(t^{-1}s)\;d\mu(t)\\
f^*(s) & = & \Delta_G(s^{-1})\overline{f(s^{-1})}\\
\|f\|_1 & = & \int_G |f|\; d\mu
\end{eqnarray*}
With this operations and norm, $C_c(G)$ has a normed *-algebra \PMlinkescapetext{structure} and $L^1(G)$ can be defined as its completion.
%%%%%
%%%%%
\end{document}
