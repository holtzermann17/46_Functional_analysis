\documentclass[12pt]{article}
\usepackage{pmmeta}
\pmcanonicalname{GeneralizedFarkasLemma}
\pmcreated{2013-03-22 17:20:53}
\pmmodified{2013-03-22 17:20:53}
\pmowner{stevecheng}{10074}
\pmmodifier{stevecheng}{10074}
\pmtitle{generalized Farkas lemma}
\pmrecord{7}{39705}
\pmprivacy{1}
\pmauthor{stevecheng}{10074}
\pmtype{Theorem}
\pmcomment{trigger rebuild}
\pmclassification{msc}{46A03}
\pmclassification{msc}{46A20}
\pmclassification{msc}{15A39}
\pmclassification{msc}{49J35}
\pmsynonym{Farkas lemma for topological vector spaces}{GeneralizedFarkasLemma}
\pmsynonym{generalized Farkas theorem}{GeneralizedFarkasLemma}
\pmrelated{AntiCone}
\pmrelated{Cone5}

% The standard font packages
\usepackage{amssymb}
\usepackage{amsmath}
\usepackage{amsfonts}

% For neatly defining theorems and definitions
\usepackage{amsthm}

% Including EPS/PDF graphics (\includegraphics)
%\usepackage{graphicx}

% Making matrix-based graphics
%%%\usepackage{xypic}

% Enumeration lists with different styles
\usepackage{enumerate}

% Set up the theorem environments
\newtheorem{thm}{Theorem}
\newtheorem*{thm*}{Theorem}

\newcommand{\defnterm}[1]{\emph{#1}}

% The standard number systems
\newcommand{\complex}{\mathbb{C}}
\newcommand{\real}{\mathbb{R}}
\newcommand{\rat}{\mathbb{Q}}
\newcommand{\nat}{\mathbb{N}}
\newcommand{\intset}{\mathbb{Z}}

% Absolute values and norms
% Normal, wide, and big versions of the delimeters
\newcommand{\abs}[1]{\lvert#1\rvert}
\newcommand{\absW}[1]{\left\lvert#1\right\rvert}
\newcommand{\absB}[1]{\Bigl\lvert#1\Bigr\rvert}
\newcommand{\norm}[1]{\lVert#1\rVert}
\newcommand{\normW}[1]{\left\lVert#1\right\rVert}
\newcommand{\normB}[1]{\Bigl\lVert#1\Bigr\rVert}

% Inverse functions
\newcommand{\inv}[1]{{#1}^{-1}}

% Differentiation operators
\newcommand{\od}[2]{\frac{d #1}{d #2}}
\newcommand{\pd}[2]{\frac{\partial #1}{\partial #2}}
\newcommand{\pdd}[2]{\frac{\partial^2 #1}{\partial #2}}
\newcommand{\ipd}[2]{\partial #1 / \partial #2}

% Differentials on integrals
\newcommand{\dx}{\, dx}
\newcommand{\dt}{\, dt}
\newcommand{\dmu}{\, d\mu}

% Inner products
\newcommand{\ip}[2]{\langle {#1}, {#2} \rangle}

% Complex numbers
\DeclareMathOperator{\zRe}{Re}
\DeclareMathOperator{\zIm}{Im}
\newcommand{\conjug}[1]{\overline{#1}}

% Calligraphic letters
\newcommand{\sF}{\mathcal{F}}
\newcommand{\sD}{\mathcal{D}}

% Standard spaces
\newcommand{\Hilb}{\mathcal{H}}
\newcommand{\Le}{\mathbf{L}}

% Operators and functions occassionally used in my articles
\DeclareMathOperator{\D}{D}
\DeclareMathOperator{\linspan}{span}
\DeclareMathOperator{\rank}{rank}
\DeclareMathOperator{\lindim}{dim}
\DeclareMathOperator{\sinc}{sinc}

% Probability stuff
\newcommand{\PP}{\mathbb{P}}
\newcommand{\E}{\mathbb{E}}

\begin{document}
\PMlinkescapeword{ways}


\emph{Farkas' Lemma} of convex optimization
and linear programming can be formulated 
for topological vector spaces.  

The more abstract version of Farkas' Lemma is useful
for understanding the essence of the usual version
of the lemma proven for matrices,
and of course, for solving optimization problems
in infinite-dimensional spaces.

The key insight
is that the notion of linear inequalities in
a finite number of real variables
can be generalized to abstract linear spaces
by the concept of a cone.


\subsection{Formal statements}

Farkas' Lemma may be stated in the several equivalent ways.
Theorem 1 is conceptually the simplest,
but Theorem 2 and 3 are more convenient for applications.

\begin{thm}
Let $X$ be a real vector spaces, and $X'$ be a subspace of linear 
functionals on $X$ that separate points.
Impose on $X$ the weak topology generated by $X'$.

Given $x \in X$, 
and a weakly-closed convex cone $K \subseteq X$,
the following are equivalent:
\begin{enumerate}[(a)]
\item
$x \in K$.
\item
If $\phi \in X'$ satisfies $\phi(y) \geq 0$ for all $y \in K$,
then $\phi(x) \geq 0$.
\item
$\phi(x) \geq 0$ for all $\phi \in K^+$ (anti-cone of $K$ with respect to $X'$).
\end{enumerate}
\begin{proof}
The equivalence of conditions (a) and (c) is a fundamental property
of the anti-cone, while condition (b) is merely
a rephrasal of condition (c).
\end{proof}
\end{thm}


Theorem 2 is a version of Theorem 1 where
the vector space and its dual space switch roles.

\begin{thm}
Let $X$ be a real vector space, and $X'$ 
be a subspace of linear functionals on 
$X$ that separate points. 
Impose on $X'$ the weak-* topology generated by $X$.

Given a functional $f \in X'$, and a weak-* closed convex cone 
$K \subseteq X'$, 
the following are equivalent:
\begin{enumerate}[(a)]
\item
$f \in K$.
\item
$\{ x \in X\colon f(x) \geq 0\} \supseteq \bigcap_{\phi \in K} \{ x \in X \colon
 \phi(x) \geq 0 \}$.
\end{enumerate}

\begin{proof}
Make the substitutions $X \to X'$, $X' \to X$, $x \to f$
and $K \to K$ in Theorem 1.
\end{proof}
\end{thm}

Theorem 3 incorporates inequalities defined
by linear mappings; such linear mappings are the analogues to the matrices
involved in the finite-dimensional version of Farkas' Lemma.

\begin{thm}
Let $X$ and $Y$ be real vector spaces,
with corresponding spaces of linear functionals $X'$ and $Y'$
that separate points.
Have $X'$ and $Y'$ generate the weak topology for $X$ and $Y$
respectively.

Given $y \in Y$, 
a linear mapping $T\colon X \to Y$, and
a subset $K \subseteq X$ such that $T(K)$ is a weakly-closed convex cone,
the following are equivalent:
\begin{enumerate}[(a)]
\item
The linear equation $Tx = y$ has a solution $x \in K$.
\item
If $\psi \in Y'$ satisfies $\psi(Tx) \geq 0$ for all $x \in K$,
then $\psi(y) \geq 0$.
\item
If $\psi \in Y'$ satisfies
$T^* \psi \in K^+$ (anti-cone of $K$ with respect to $X'$), 
then $\psi(y) \geq 0$.
\end{enumerate}
Here $T^* \colon Y' \to X'$ denotes the pullback, restricted to $Y'$ and $X'$,
defined by $T^* \psi = \psi \circ T$.
\begin{proof}
Make the substitutions $X \to Y$, $X' \to Y'$, $x \to y$ and $K \to T(K)$ in Theorem 1.
Condition (c) is a rephrasal of condition (b).
\end{proof}
\end{thm}

\begin{thebibliography}{6}
\bibitem{CK}
B. D. Craven and J. J. Kohila.
``Generalizations of Farkas' Theorem.''
\emph{SIAM Journal on Mathematical Analysis}. 
Vol. 8, No. 6, November 1977.
\bibitem{KC}
David Kincaid and Ward Cheney. 
\emph{Numerical Analysis: Mathematics of Scientific Computing},
third edition. Brooks/Cole, 2002.
\end{thebibliography}

%%%%%
%%%%%
\end{document}
