\documentclass[12pt]{article}
\usepackage{pmmeta}
\pmcanonicalname{BibliographyForMathematicalBiophysicsAndRelationalTheories}
\pmcreated{2013-03-22 18:18:18}
\pmmodified{2013-03-22 18:18:18}
\pmowner{bci1}{20947}
\pmmodifier{bci1}{20947}
\pmtitle{bibliography for mathematical biophysics and relational theories}
\pmrecord{15}{40925}
\pmprivacy{1}
\pmauthor{bci1}{20947}
\pmtype{Bibliography}
\pmcomment{trigger rebuild}
\pmclassification{msc}{46L55}
\pmclassification{msc}{92B20}
\pmclassification{msc}{92B05}
\pmclassification{msc}{37N25}
\pmclassification{msc}{92B99}
\pmclassification{msc}{92-00}
\pmsynonym{theoretical biophysics bibliography}{BibliographyForMathematicalBiophysicsAndRelationalTheories}
\pmsynonym{relational theories in biology}{BibliographyForMathematicalBiophysicsAndRelationalTheories}
%\pmkeywords{Bibliography for mathematical biophysics}
%\pmkeywords{and relational theories}
%\pmkeywords{mathematical biology}
%\pmkeywords{theoretical biology}
%\pmkeywords{theoretical biophysics}
%\pmkeywords{bioinformatics}
%\pmkeywords{quantum genetics}
%\pmkeywords{molecular biology theories}
\pmrelated{MathematicalBiology}
\pmrelated{TopicEntryOnAppliedMathematics}
\pmrelated{OverviewOfTheContentOfPlanetMath}
\pmrelated{PhysicalMathematicsAndEngineeringTopicOnAppliedMathematicalPhysics}

\endmetadata

% this is the dehttp://planetmath.org/?op=addenfault PlanetMath preamble. 
\usepackage{amssymb}
\usepackage{amsmath}
\usepackage{amsfonts}
%%%\usepackage{xypic}
% define commands here
\usepackage{amsmath, amssymb, amsfonts, amsthm, amscd, latexsym}
%%\usepackage{xypic}
\usepackage[mathscr]{eucal}

\setlength{\textwidth}{6.5in}
%\setlength{\textwidth}{16cm}
\setlength{\textheight}{9.0in}
%\setlength{\textheight}{24cm}

\hoffset=-.75in     %%ps format
%\hoffset=-1.0in     %%hp format
\voffset=-.4in

\theoremstyle{plain}
\newtheorem{lemma}{Lemma}[section]
\newtheorem{proposition}{Proposition}[section]
\newtheorem{theorem}{Theorem}[section]
\newtheorem{corollary}{Corollary}[section]

\theoremstyle{definition}
\newtheorem{definition}{Definition}[section]
\newtheorem{example}{Example}[section]
%\theoremstyle{remark}
\newtheorem{remark}{Remark}[section]
\newtheorem*{notation}{Notation}
\newtheorem*{claim}{Claim}

\renewcommand{\thefootnote}{\ensuremath{\fnsymbol{footnote%%@
}}}
\numberwithin{equation}{section}

\newcommand{\Ad}{{\rm Ad}}
\newcommand{\Aut}{{\rm Aut}}
\newcommand{\Cl}{{\rm Cl}}
\newcommand{\Co}{{\rm Co}}
\newcommand{\DES}{{\rm DES}}
\newcommand{\Diff}{{\rm Diff}}
\newcommand{\Dom}{{\rm Dom}}
\newcommand{\Hol}{{\rm Hol}}
\newcommand{\Mon}{{\rm Mon}}
\newcommand{\Hom}{{\rm Hom}}
\newcommand{\Ker}{{\rm Ker}}
\newcommand{\Ind}{{\rm Ind}}
\newcommand{\IM}{{\rm Im}}
\newcommand{\Is}{{\rm Is}}
\newcommand{\ID}{{\rm id}}
\newcommand{\GL}{{\rm GL}}
\newcommand{\Iso}{{\rm Iso}}
\newcommand{\Sem}{{\rm Sem}}
\newcommand{\St}{{\rm St}}
\newcommand{\Sym}{{\rm Sym}}
\newcommand{\SU}{{\rm SU}}
\newcommand{\Tor}{{\rm Tor}}
\newcommand{\U}{{\rm U}}

\newcommand{\A}{\mathcal A}
\newcommand{\Ce}{\mathcal C}
\newcommand{\D}{\mathcal D}
\newcommand{\E}{\mathcal E}
\newcommand{\F}{\mathcal F}
\newcommand{\G}{\mathcal G}
\newcommand{\Q}{\mathcal Q}
\newcommand{\R}{\mathcal R}
\newcommand{\cS}{\mathcal S}
\newcommand{\cU}{\mathcal U}
\newcommand{\W}{\mathcal W}

\newcommand{\bA}{\mathbb{A}}
\newcommand{\bB}{\mathbb{B}}
\newcommand{\bC}{\mathbb{C}}
\newcommand{\bD}{\mathbb{D}}
\newcommand{\bE}{\mathbb{E}}
\newcommand{\bF}{\mathbb{F}}
\newcommand{\bG}{\mathbb{G}}
\newcommand{\bK}{\mathbb{K}}
\newcommand{\bM}{\mathbb{M}}
\newcommand{\bN}{\mathbb{N}}
\newcommand{\bO}{\mathbb{O}}
\newcommand{\bP}{\mathbb{P}}
\newcommand{\bR}{\mathbb{R}}
\newcommand{\bV}{\mathbb{V}}
\newcommand{\bZ}{\mathbb{Z}}

\newcommand{\bfE}{\mathbf{E}}
\newcommand{\bfX}{\mathbf{X}}
\newcommand{\bfY}{\mathbf{Y}}
\newcommand{\bfZ}{\mathbf{Z}}

\renewcommand{\O}{\Omega}
\renewcommand{\o}{\omega}
\newcommand{\vp}{\varphi}
\newcommand{\vep}{\varepsilon}

\newcommand{\diag}{{\rm diag}}
\newcommand{\grp}{{\mathbb G}}
\newcommand{\dgrp}{{\mathbb D}}
\newcommand{\desp}{{\mathbb D^{\rm{es}}}}
\newcommand{\Geod}{{\rm Geod}}
\newcommand{\geod}{{\rm geod}}
\newcommand{\hgr}{{\mathbb H}}
\newcommand{\mgr}{{\mathbb M}}
\newcommand{\ob}{{\rm Ob}}
\newcommand{\obg}{{\rm Ob(\mathbb G)}}
\newcommand{\obgp}{{\rm Ob(\mathbb G')}}
\newcommand{\obh}{{\rm Ob(\mathbb H)}}
\newcommand{\Osmooth}{{\Omega^{\infty}(X,*)}}
\newcommand{\ghomotop}{{\rho_2^{\square}}}
\newcommand{\gcalp}{{\mathbb G(\mathcal P)}}

\newcommand{\rf}{{R_{\mathcal F}}}
\newcommand{\glob}{{\rm glob}}
\newcommand{\loc}{{\rm loc}}
\newcommand{\TOP}{{\rm TOP}}

\newcommand{\wti}{\widetilde}
\newcommand{\what}{\widehat}

\renewcommand{\a}{\alpha}
\newcommand{\be}{\beta}
\newcommand{\ga}{\gamma}
\newcommand{\Ga}{\Gamma}
\newcommand{\de}{\delta}
\newcommand{\del}{\partial}
\newcommand{\ka}{\kappa}
\newcommand{\si}{\sigma}
\newcommand{\ta}{\tau}
\newcommand{\med}{\medbreak}
\newcommand{\medn}{\medbreak \noindent}
\newcommand{\bign}{\bigbreak \noindent}
\newcommand{\lra}{{\longrightarrow}}
\newcommand{\ra}{{\rightarrow}}
\newcommand{\rat}{{\rightarrowtail}}
\newcommand{\oset}[1]{\overset {#1}{\ra}}
\newcommand{\osetl}[1]{\overset {#1}{\lra}}
\newcommand{\hr}{{\hookrightarrow}}
\begin{document}
This is a bibliography for areas of applied mathematics concerned with mathematical/relational and physical modeling and mathematical applications to life sciences,complex systems/complex systems biology and medicine.

\subsection{A Bibliography for Mathematical Biophysics, Mathematical Biology and Theoretical Biology}


\begin{thebibliography}{9}

\bibitem{ES45}
Erwin Schr\"odinger.1945. \emph{What is Life?}. Cambridge University Press: Cambridge (UK).

\bibitem{NR54}
Nicolas Rashevsky.1954, Topology and life: In search of general mathematical principles in 
biology and sociology, \emph{Bull. Math. Biophys.} 16: 317-348. 

\bibitem{NR65}
Nicolas Rashevsky. 1965. Models and Mathematical Principles in Biology. In: Waterman/Morowitz, \emph{Theoretical and Mathematical Biology}, pp. 36-53.

\bibitem{REF53}
Rosalind E. Franklin and R.G. Gosling. 1953. Evidence for 2-chain helix in crystalline structure
of sodium deoxyribonucleate (DNA). \emph{Nature} 177: 928-930. 

\bibitem{WM-SW-SR-HRW53}
Wilkins, M.H.F. et al. 1953. Helical structure of crystalline deoxypentose nucleic acid (DNA).
\emph{Nature} 172: 759-762.

\bibitem{FHCC53} 
Francis H.C. Crick. 1953. Fourier transform of a coiled coil. \emph{Acta Cryst}. 6: 685-687

\bibitem{HRW68}
H. R. Wilson. 1966. \emph{Diffraction of X-rays by Proteins, Nucleic Acids and Viruses}.
London: Arnold.

\bibitem{BBGG06}
I. C. Baianu, J. F. Glazebrook, R. Brown and G. Georgescu.: Complex Nonlinear Biodynamics in Categories, Higher dimensional Algebra and \L{}ukasiewicz-Moisil Topos: Transformation of Neural, Genetic and Neoplastic Networks, \emph{Axiomathes}, \textbf{16}: 65-122 2006).
\PMlinkexternal{PDF file of document}{http://www.bangor.ac.uk/~mas010/pdffiles/Axio7complx_Printedk7_v17p223_fulltext.pdf}

\bibitem{ICB78}
I.C. Baianu. 1978. X-ray Scattering by Partially Disordered Membrane Lattices. {\em Acta Crystall}. A34: 731-753.   
(\emph{paper contributed from The Cavendish Laboratory, Cambridge in 1979}). 

\bibitem{ICB80}
I.C. Baianu. 1980. Structural Order and Partial Disorder in Biological Systems. \emph{Bull. Math. Biol.}
(\emph{paper contributed from The Cavendish Laboratory, Cambridge in 1979}). 

\bibitem{RH-SNB62}
R. Hosemann and S. N. Bagchi. 1962. \emph{Direct Analysis of Diffraction by Matter}. Amsterdam: North Holland.

\bibitem{VoetD-JG95}
D. Voet and J.G. Voet. 1995. \emph{Biochemistry}. 2nd Edition, New York, Chichester, Brisbone, Toronto,
Singapore: J. Wiley and Sons, INC., 1361 pp.. (\emph{an excellently illustrated textbook})

\bibitem{RR97}
Robert Rosen. 1997 and 2002. \emph{Essays on Life Itself}. 

\bibitem{RRosen1}
Rosen, R.: 1958a, A Relational Theory of Biological Systems 
\emph{Bulletin of Mathematical Biophysics} \textbf{20}: 245-260.

\bibitem{RRosen2}
Rosen, R.: 1958b, The Representation of Biological Systems from the Standpoint of the Theory of Categories., 
\emph{Bulletin of Mathematical Biophysics} \textbf{20}: 317-341.

\bibitem{RRosen60}
Rosen, R. 1960. A quantum-theoretic approach to genetic problems. \emph{Bulletin of Mathematical Biophysics} 
22: 227-255.

\bibitem{RRosen87}
Rosen, R.: 1987, On Complex Systems, \emph{European Journal of Operational Research} 
\textbf{30}, 129-134.

\bibitem{RR70}
Rosen,R. 1970, \emph{Dynamical Systems Theory in Biology}. New York: Wiley Interscience. 

\bibitem{RR70}
Rosen,R. 1970, \emph{Optimality Principles in Biology}, New York and London: Academic Press. 

\bibitem{RR70}
Rosen,R. 1978, \emph{Fundamentals of Measurement and Representation of Natural Systems}, Elsevier Science Ltd, 

\bibitem{RR70}
Rosen,R. 1985, \emph{Anticipatory Systems: Philosophical, Mathematical and Methodological Foundations}. Pergamon Press. 

\bibitem{RR91}
Rosen,R. 1991, \emph{Life Itself: A Comprehensive Inquiry into the Nature, Origin, and Fabrication of Life}, Columbia University Press 

\bibitem{EC84}
Ehresmann, C.: 1984, \emph{Oeuvres compl\`etes et  comment\'ees: Amiens, 1980-84}, edited and commented 
by Andr\'ee Ehresmann.

\bibitem{EACV2}
Ehresmann, A. C. and J.-P. Vanbremersch: 2006, The Memory Evolutive Systems as a Model of Rosen's Organisms, 
in \emph{Complex Systems Biology}, I.C. Baianu, Editor, \emph{Axiomathes} \textbf{16} (1--2), pp. 13-50.

\bibitem{EML1}
Eilenberg, S. and Mac Lane, S.: 1942, Natural Isomorphisms in Group Theory., \emph{American Mathematical Society 43}: 757-831.

\bibitem{EL}
Eilenberg, S. and Mac Lane, S.: 1945, The General Theory of Natural Equivalences, 
\emph{Transactions of the American Mathematical Society} 58: 231-294.

\bibitem{Elsasser}
Elsasser, M.W.: 1981, A Form of Logic Suited for Biology., In: Robert, Rosen, ed., \emph{Progress in Theoretical Biology},  Volume 6, Academic Press, New York and London, pp 23-62.

\bibitem{BAF60}
Bartholomay, A. F.: 1960. Molecular Set Theory. A mathematical representation for chemical reaction mechanisms. \emph{Bull. Math. Biophys.}, \textbf{22}: 285-307.

\bibitem{BAF65}
Bartholomay, A. F.: 1965. Molecular Set Theory: II. An aspect of biomathematical theory of sets., \emph{Bull. Math. Biophys.} \textbf{27}: 235-251.

\bibitem{BAF71}
Bartholomay, A.: 1971. Molecular Set Theory: III. The Wide-Sense Kinetics of Molecular Sets ., \emph{Bulletin of Mathematical Biophysics}, \textbf{33}: 355-372.

\end{thebibliography}
%%%%%
%%%%%
\end{document}
