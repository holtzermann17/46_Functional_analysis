\documentclass[12pt]{article}
\usepackage{pmmeta}
\pmcanonicalname{MinimalUnitizationsOfAlgebrasWithAdditionalStructure}
\pmcreated{2013-03-22 17:46:29}
\pmmodified{2013-03-22 17:46:29}
\pmowner{asteroid}{17536}
\pmmodifier{asteroid}{17536}
\pmtitle{minimal unitizations of algebras with additional structure}
\pmrecord{6}{40232}
\pmprivacy{1}
\pmauthor{asteroid}{17536}
\pmtype{Result}
\pmcomment{trigger rebuild}
\pmclassification{msc}{46L05}
\pmclassification{msc}{46K05}
\pmclassification{msc}{46H05}
\pmclassification{msc}{16W99}
\pmclassification{msc}{16B99}
\pmdefines{minimal unitization of a topological algebra}
\pmdefines{minimal unitization of a Banach algebra}
\pmdefines{minimal unitization of a Banach-* algebra}
\pmdefines{minimal unitization of a $C^*$-algebra}

\endmetadata

% this is the default PlanetMath preamble.  as your knowledge
% of TeX increases, you will probably want to edit this, but
% it should be fine as is for beginners.

% almost certainly you want these
\usepackage{amssymb}
\usepackage{amsmath}
\usepackage{amsfonts}

% used for TeXing text within eps files
%\usepackage{psfrag}
% need this for including graphics (\includegraphics)
%\usepackage{graphicx}
% for neatly defining theorems and propositions
%\usepackage{amsthm}
% making logically defined graphics
%%%\usepackage{xypic}

% there are many more packages, add them here as you need them

% define commands here

\begin{document}
\PMlinkescapeword{structure}
\PMlinkescapeword{algebra}

Given a (non-unital) \PMlinkname{algebra}{Algebra} there is a \PMlinkescapetext{minimal} procedure to add an unit to it (\PMlinkname{parent entry}{Unitization}). When the algebra has some additional structure (topological structure, for example), it is often useful to endow the same structure on the minimal unitization of the algebra.

All the algebras are to be considered non-unital.


\subsection{Topological Algebras}

Let $\mathcal{A}$ be a topological algebra algebra over a (topological) field $\mathbb{K}$. Let $\widetilde{\mathcal{A}}$ be its minimal unitization.

Then $\widetilde{\mathcal{A}} = \mathcal{A} \oplus \mathbb{K}$ is a topological algebra with the product topology.


\subsection{Normed and Banach Algebras}
Let $\mathcal{A}$ be a normed algebra over $\mathbb{K}$ ($= \mathbb{R}$ or $\mathbb{C}$) with norm $\| \cdot \|$. Let $\widetilde{\mathcal{A}}$ be its minimal unitization.

Then $\widetilde{\mathcal{A}}$ is a normed algebra under the norm $\| \cdot \|_u$:

\begin{displaymath}
\|a+\lambda\|_u = \|a\| +|\lambda|\,, \qquad a \in \mathcal{A} \;, \lambda \in \mathbb{K}
\end{displaymath}

Moreover, if $\mathcal{A}$ is a Banach algebra, then $\widetilde{\mathcal{A}}$ is a Banach algebra with the norm $\|\cdot\|_u$.

\subsection{*-algebras}

Let $\mathcal{A}$ be a *-algebra over an \PMlinkname{involutory field}{InvolutaryRing} $\mathbb{K}$. Let $\widetilde{\mathcal{A}}$ be its minimal unitization.

Then $\widetilde{\mathcal{A}}$ is a *-algebra with involution given by:

\begin{displaymath}
(a+\lambda)^* = a^* + \overline{\lambda}\, \qquad a \in \mathcal{A} ,\; \lambda \in \mathbb{K}
\end{displaymath}

\subsection{Topological *-algebras, Normed *-algebras and Banach *-algebras}

Let $\mathcal{A}$ be a topological *-algebra over $\mathbb{C}$. Let $\widetilde{\mathcal{A}}$ be its minimal unitization.

Then $\widetilde{\mathcal{A}}$ is a topological *-algebra with the product topology and the involution defined above.

Also, if $\mathcal{A}$ is a normed *-algebra (Banach -*algebra), then $\widetilde{\mathcal{A}}$ is also a normed *-algebra (Banach *-algebra) under the above involution and the norm $\|\cdot\|_u$.

\subsection{C*-algebras}

Let $\mathcal{A}$ be a \PMlinkname{$C^*$-algebra}{CAlgebra} with norm $\| \cdot \|$. Let $\widetilde{\mathcal{A}}$ be its minimal unitization.

Then $\widetilde{\mathcal{A}}$ is $C^*$-algebra under the norm $\| \cdot \|_L$:

\begin{displaymath}
\|a+\lambda\|_L = \sup_{\|b\|=1}\|ab +\lambda b\|\,, \qquad a \in \mathcal{A} ,\; \lambda \in \mathbb{C}
\end{displaymath}

This norm comes from regarding elements of $\widetilde{\mathcal{A}}$ as left \PMlinkescapetext{multiplication operators} on $\mathcal{A}$. The norm $\| \cdot \|_L$ is \PMlinkescapetext{equivalent} to the norm $\| \cdot \|_u$.
%%%%%
%%%%%
\end{document}
