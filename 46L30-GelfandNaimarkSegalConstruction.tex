\documentclass[12pt]{article}
\usepackage{pmmeta}
\pmcanonicalname{GelfandNaimarkSegalConstruction}
\pmcreated{2013-03-22 17:47:40}
\pmmodified{2013-03-22 17:47:40}
\pmowner{asteroid}{17536}
\pmmodifier{asteroid}{17536}
\pmtitle{Gelfand-Naimark-Segal construction}
\pmrecord{10}{40256}
\pmprivacy{1}
\pmauthor{asteroid}{17536}
\pmtype{Feature}
\pmcomment{trigger rebuild}
\pmclassification{msc}{46L30}
\pmclassification{msc}{46L05}
\pmsynonym{GNS construction}{GelfandNaimarkSegalConstruction}
%\pmkeywords{GNS pair}
%\pmkeywords{GNS representation}
\pmrelated{CAlgebra}
\pmrelated{CAlgebra3}
\pmrelated{RepresentationOfAC_cG_dTopologicalAlgebra}
\pmdefines{GNS pair}
\pmdefines{GNS representation}
\pmdefines{pure states and irreducible representations}

% this is the default PlanetMath preamble.  as your knowledge
% of TeX increases, you will probably want to edit this, but
% it should be fine as is for beginners.

% almost certainly you want these
\usepackage{amssymb}
\usepackage{amsmath}
\usepackage{amsfonts}

% used for TeXing text within eps files
%\usepackage{psfrag}
% need this for including graphics (\includegraphics)
%\usepackage{graphicx}
% for neatly defining theorems and propositions
%\usepackage{amsthm}
% making logically defined graphics
%%%\usepackage{xypic}

% there are many more packages, add them here as you need them

% define commands here

\begin{document}
\PMlinkescapeword{properties}
\PMlinkescapeword{bounded}
\PMlinkescapeword{dense in}
\PMlinkescapeword{irreducible}
\PMlinkescapeword{cyclic}

\section{GNS Construction}

The Gelfand-Naimark-Segal construction (or GNS construction) is a fundamental idea in the \PMlinkescapetext{theory} of \PMlinkescapetext{operator} \PMlinkescapetext{algebras}. It provides a procedure to construct and study representations of \PMlinkname{$C^*$-algebras}{CAlgebra} and is the first step on the proof of the Gelfand-Naimark representation theorem, which \PMlinkescapetext{states} that every $C^*$-algebra is isometrically isomorphic to a closed *-subalgebra of $B(H)$, the algebra of bounded operators on a Hilbert space $H$.

There are generalizations of this construction for Banach *-algebras with an approximate unit, and some of the results stated here are in fact valid for this kind of algebras, but we will restrict our attention to the $C^*$ case.

\section{Representations associated with positive linear functionals}

Let $\mathcal{A}$ be a $C^*$-algebra and $\phi$ a positive linear functional in $\mathcal{A}$.

We are going to construct a representation $\pi_{\phi}$ of $\mathcal{A}$ and for that we need to construct a suitable Hilbert space.

Let us endow $\mathcal{A}$ with a semi-inner product defined by $\langle x, y \rangle_{\phi} : = \phi(y^*x)$. Now we define the set
\begin{displaymath}
N_{\phi} : = \{ x \in \mathcal{A}: \langle x , x \rangle_{\phi}=0 \}
\end{displaymath}
It is easily seen that $N_{\phi}$ is a closed \PMlinkname{left ideal}{Ideal} in $\mathcal{A}$ (using the Cauchy-Schwarz inequality, which is valid in semi-inner product spaces), so that $\langle \cdot, \cdot \rangle_{\phi}$ induces a well defined inner product on the \PMlinkname{quotient}{QuotientModule} $\mathcal{A}/N_{\phi}$. The completion of $\mathcal{A}/N_{\phi}$ is then an Hilbert space, which we will be denoted by $H_{\phi}$.

We will now define a representation of $\mathcal{A}$ on $H_{\phi}$ by left multiplication. For every $a \in \mathcal{A}$ let $\pi_{\phi}(a)$ be the operator of left multiplication by $a$ on $\mathcal{A}/N_{\phi}$, i.e.
\begin{displaymath}
\pi(a)\,(x+ N_{\phi}) := ax+N_{\phi}
\end{displaymath}

$\,$

{\bf Theorem 1 -} The function $\pi_{\phi}(a):\mathcal{A}/N_{\phi} \longrightarrow \mathcal{A}/N_{\phi}$ defined above is linear and \PMlinkname{bounded}{BoundedOperator}, with $\|\pi_{\phi}(a)\| \leq \|a\|$.

$\;$

Being bounded, the operator $\pi_{\phi}(a)$ extends uniquely to a bounded operator on $H_{\phi}$, which we denote by the same symbol, $\pi_{\phi}(a)$.

Let $B(H_{\phi})$ be the algebra of bounded operators on $H_{\phi}$.

$\,$

{\bf Theorem 2 -} The function $\pi_{\phi}:\mathcal{A} \longrightarrow B(H_{\phi})$ defined by $a \mapsto \pi_{\phi}(a)$ is a $C^*$-algebra representation of $\mathcal{A}$.

This representation is called the {\bf GNS representation} associated to $\phi$.

\section{Cyclic vectors and GNS pairs}

Suppose $\mathcal{A}$ had an identity element $e$. In this case it is easily seen that there exists a cyclic vector $\xi_{\phi} \in H_{\phi}$, i.e. a vector $\xi_{\phi}$ such that $\pi_{\phi}(\mathcal{A})\,\xi_{\phi}$ is \PMlinkname{dense}{Dense} in $H_{\phi}$. This cyclic vector can just be chosen as $e + N_{\phi}$.

Moreoever, this cyclic vector $\xi_{\phi} :=e + N_{\phi}$ is such that $\phi(a) = \langle \pi_{\phi}(a)\,\xi_{\phi}, \xi_{\phi}\rangle_{\phi}$ for every $a \in \mathcal{A}$.

Thus, in this case the representation $\pi_{\phi}$ is \PMlinkname{cyclic}{BanachAlgebraRepresentation} and $\phi$ is a vector state of $\mathcal{A}$. The result is still valid for general $C^*$-algebras:

$\,$

{\bf Theorem 3 -} Let $\pi_{\phi}$ be the representation of $\mathcal{A}$ defined previously. Then there exists a vector $\xi_{\phi} \in H_{\phi}$ such that
\begin{itemize}
\item $\pi_{\phi}(\mathcal{A})\,\xi_{\phi}$ is dense in $H_{\phi}$, i.e. $\pi_{\phi}$ is cyclic,
\item $\phi(a) = \langle \pi_{\phi}(a)\,\xi_{\phi}, \xi_{\phi}\rangle_{\phi}$ for every $a \in \mathcal{A}$, i.e. $\phi$ is a vector state.
\end{itemize}

$\,$

Any pair $(\pi, \xi)$, where $\pi$ is a representation of $\mathcal{A}$ on a Hilbert space $H$ and $\xi \in H$, satisfying the above conditions for $\phi$:
\begin{itemize}
\item $\pi(\mathcal{A})\,\xi$ is dense in $H$,
\item $\phi(a) = \langle \pi(a)\,\xi, \xi\rangle$ for every $a \in \mathcal{A}$
\end{itemize}
is called a {\bf GNS pair} for $\phi$.

$\,$

{\bf Theorem 4 -} All GNS pairs for $\phi$ are \PMlinkescapetext{equivalent} (in the sense that the corresponding representations are unitarily equivalent).


\section{Irreducible representations}

We know that are "plenty" of states on $C^*$-algebra (see \PMlinkname{this entry}{PropertiesOfStates}), and so we have assured the existence of many (cyclic) representations. An interesting fact is that this representations associated to states are \PMlinkname{irreducible}{BanachAlgebraRepresentation} exactly when the state is a pure state:

$\,$

{\bf Theorem 5 -} Let $\phi$ be a state on $\mathcal{A}$. Then the representation $\pi_{\phi}$ is irreducible if and only if $\phi$ is a pure state.

$\,$

The fact that there are "plenty" of pure states in a $C^*$-algebra allows one to assure the existence of irreducible representations that preserve the norm of a given element in $\mathcal{A}$.

$\,$

{\bf Theorem 6 -} Let $\mathcal{A}$ be a $C^*$-algebra. For every element $a$ there exists an irreducible representation $\pi$ of $\mathcal{A}$ such that $\|\pi(a)\| = \|a\|$.

$\,$

This last theorem is a fundamental step in the proof of the Gelfand-Naimark representation theorem.
%%%%%
%%%%%
\end{document}
