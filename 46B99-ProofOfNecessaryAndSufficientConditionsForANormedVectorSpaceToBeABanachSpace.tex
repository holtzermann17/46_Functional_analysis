\documentclass[12pt]{article}
\usepackage{pmmeta}
\pmcanonicalname{ProofOfNecessaryAndSufficientConditionsForANormedVectorSpaceToBeABanachSpace}
\pmcreated{2013-03-22 17:35:11}
\pmmodified{2013-03-22 17:35:11}
\pmowner{willny}{13192}
\pmmodifier{willny}{13192}
\pmtitle{proof of necessary and sufficient conditions for a normed vector space to be a Banach space}
\pmrecord{4}{39998}
\pmprivacy{1}
\pmauthor{willny}{13192}
\pmtype{Proof}
\pmcomment{trigger rebuild}
\pmclassification{msc}{46B99}

\endmetadata

% this is the default PlanetMath preamble.  as your knowledge
% of TeX increases, you will probably want to edit this, but
% it should be fine as is for beginners.

% almost certainly you want these
\usepackage{amssymb}
\usepackage{amsmath}
\usepackage{amsfonts}

% used for TeXing text within eps files
%\usepackage{psfrag}
% need this for including graphics (\includegraphics)
%\usepackage{graphicx}
% for neatly defining theorems and propositions
%\usepackage{amsthm}
% making logically defined graphics
%%%\usepackage{xypic}

% there are many more packages, add them here as you need them

% define commands here

\begin{document}
We prove here that in order for a normed space, say $X$, with the norm, say $\|\cdot\|$ to be
Banach, it is necessary and sufficient that convergence of every absolutely convergent series in $X$
implies convergence of the series in $X$.
\\
\\Suppose that $X$ is Banach. Let a sequence $(x_n)$ be in $X$ such that the series
$$\sum_n{\|x_n\|}$$
converges. Then for all $\epsilon > 0$ there exists $N$ such that for all $m > n > N$ we have
$$\left\|\sum_{n+1}^m{x_n}\right\|\leq\sum_{n+1}^m{\|x_n\|} < \epsilon$$
Hence
$$s_k = \sum_{n=1}^k{x_n}$$ is a Cauchy sequence in $X$. Since $X$ is Banach, $s_k$ converges in $X$.
\\
\\Conversely, suppose that absolute convergence implies convergence. Let $(x_n)$ be a Cauchy sequence
in $X$. Then for all $m \geq 1$ there exists $N_m$ such that for all $k, k' \geq N_m$ we have $\|x_k - x_{k'}\| < 1/m^2$.
We'll conveniently choose $N_m$ so that $N_m$ is an increasing sequence in $m$. Then in particular, $\|x_{N_m} - x_{N_{m+1}}\| < 1/m^2$. Hence we have,
$$\sum_{m = 1}^M{\|x_{N_m} - x_{N_{m+1}}\|} < \sum_{m=1}^M{\frac{1}{m^2}}$$
The sum on the right converges, so must the sum on the left. Since absolute convergence implies convergence,
we must have
$$\sum_{m = 1}^M{(x_{N_m} - x_{N_{m+1}})}$$
converges as $M$ tends to infinity. So there is an $s$ in $X$ which is the limit of the sum above. As a telescoping
series, however, the sum above converges to $\lim_{M\rightarrow\infty}({x_{N_1} - x_{N_m}}) = s$. Since $s$ and $x_{N_1}$ are both in $X$, so is the limit of $x_{N_m}$, which is a subsequence of the Cauchy sequence $(x_n)$. Hence $(x_n)$ converges in $X$. So $X$ is Banach.
\\
\\This completes the proof.
%%%%%
%%%%%
\end{document}
