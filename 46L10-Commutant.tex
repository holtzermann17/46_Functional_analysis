\documentclass[12pt]{article}
\usepackage{pmmeta}
\pmcanonicalname{Commutant}
\pmcreated{2013-03-22 17:21:53}
\pmmodified{2013-03-22 17:21:53}
\pmowner{asteroid}{17536}
\pmmodifier{asteroid}{17536}
\pmtitle{commutant}
\pmrecord{11}{39725}
\pmprivacy{1}
\pmauthor{asteroid}{17536}
\pmtype{Definition}
\pmcomment{trigger rebuild}
\pmclassification{msc}{46L10}
\pmrelated{VonNeumannAlgebra}
\pmdefines{double commutant}

\endmetadata

% this is the default PlanetMath preamble.  as your knowledge
% of TeX increases, you will probably want to edit this, but
% it should be fine as is for beginners.

% almost certainly you want these
\usepackage{amssymb}
\usepackage{amsmath}
\usepackage{amsfonts}

% used for TeXing text within eps files
%\usepackage{psfrag}
% need this for including graphics (\includegraphics)
%\usepackage{graphicx}
% for neatly defining theorems and propositions
%\usepackage{amsthm}
% making logically defined graphics
%%%\usepackage{xypic}

% there are many more packages, add them here as you need them

% define commands here

\begin{document}
\section*{Definition}

Let $H$ be an Hilbert Space, $B(H)$ the algebra of bounded operators in $H$ and $\mathcal{F} \subset B(H)$.

The {\bf commutant} of $\mathcal{F}$, usually denoted $\mathcal{F}'$, is the subset of $B(H)$ consisting of all
 elements that commute with every element of $\mathcal{F}$, that is
\begin{center}
$\mathcal{F}'=\{T \in B(H):\; TS=ST \,,\;\;\; \forall S \in \mathcal{F}\}$
\end{center}

The {\bf double commutant} of $\mathcal{F}$ is just $(\mathcal{F}')'$ and is usually denoted $\mathcal{F}''$.

\section*{Properties:}
\begin{itemize}
\item If $\mathcal{F}_1 \subseteq \mathcal{F}_2$, then $\mathcal{F}_2' \subseteq \mathcal{F}_1'$.
\item $\mathcal{F} \subseteq \mathcal{F}''$.
\item If $\mathcal{A}$ is a subalgebra of $B(H)$, then $\mathcal{A} \cap \mathcal{A}'$ is the \PMlinkname{center}{CenterOfARing} of $\mathcal{A}$.
\item If $\mathcal{F}$ is self-adjoint then $\mathcal{F}'$ is self-adjoint.
\item $\mathcal{F}'$ is always a subalgebra of $B(H)$ that contains the identity operator and is closed in the weak operator topology.
\item If $\mathcal{F}$ is self-adjoint then $\mathcal{F}'$ is a von Neumann algebra.
\end{itemize}

{\bf Remark:} The commutant is a particular case of the more general definition of centralizer.
%%%%%
%%%%%
\end{document}
