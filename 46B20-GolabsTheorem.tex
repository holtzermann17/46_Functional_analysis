\documentclass[12pt]{article}
\usepackage{pmmeta}
\pmcanonicalname{GolabsTheorem}
\pmcreated{2013-03-22 16:50:33}
\pmmodified{2013-03-22 16:50:33}
\pmowner{Mathprof}{13753}
\pmmodifier{Mathprof}{13753}
\pmtitle{Golab's theorem}
\pmrecord{7}{39087}
\pmprivacy{1}
\pmauthor{Mathprof}{13753}
\pmtype{Theorem}
\pmcomment{trigger rebuild}
\pmclassification{msc}{46B20}

\endmetadata

% this is the default PlanetMath preamble.  as your knowledge
% of TeX increases, you will probably want to edit this, but
% it should be fine as is for beginners.

% almost certainly you want these
\usepackage{amssymb}
\usepackage{amsmath}
\usepackage{amsfonts}

% used for TeXing text within eps files
%\usepackage{psfrag}
% need this for including graphics (\includegraphics)
%\usepackage{graphicx}
% for neatly defining theorems and propositions
%\usepackage{amsthm}
% making logically defined graphics
%%%\usepackage{xypic}

% there are many more packages, add them here as you need them

% define commands here

\begin{document}
Theorem.  Let $D$ be the unit disc of a Minkowski plane and let
$\ell(\partial D)$ denote the \PMlinkname{Minkowski length}{LengthOfCurveInAMetricSpace} of the boundary of $D$. Then
$6 \leq \ell(\partial D) \leq 8$. The lower bound is attained if and only
if $D$ is linearly equivalent to a regular hexagon. The upper bound
is attained if and only if $D$ is a parallelogram.

Note that 1/2 the perimeter of the unit disc is  a constant
between 3 and 4. The special case of the 2-norm yields a constant,
which is known as $\pi$. So Golab's theorem is that "pi" for a 
Minkowski plane is always between 3 and 4.


\begin{thebibliography}{10}
\bibitem[GO]{GO}
{\scshape S. Golab},  Quelques probl\`emes m\'etriques de la g\'eometrie de Minkowski,
\it{Trav. l'Acad. Mines Cracovie} $\bold{6}$ (1932) 1-79.
\bibitem[PE]{PE}
{\scshape C.M. Petty}, Geometry of the Minkowski plane, \it{Riv. Mat. Univ. Parma} (4) $\bold{6}$ (1955) 269-292.
\bibitem[SC]{SC}
{\scshape J.J. Sch\"aefer}, Inner diameter, perimeter, and girth of spheres, 
\it{Math. Ann}. $\bold{173}$ (1967) 59-79.
\bibitem[ACT]{ACT}
{\scshape A.C. Thompson}, \it{Minkowski Geometry}, Encyclopedia of Mathematics and its Applications, 63, Cambridge University Press, Cambridge, 1996.
\end{thebibliography}




%%%%%
%%%%%
\end{document}
