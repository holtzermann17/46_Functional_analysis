\documentclass[12pt]{article}
\usepackage{pmmeta}
\pmcanonicalname{WaveletSet}
\pmcreated{2013-03-22 14:27:10}
\pmmodified{2013-03-22 14:27:10}
\pmowner{swiftset}{1337}
\pmmodifier{swiftset}{1337}
\pmtitle{wavelet set}
\pmrecord{7}{35971}
\pmprivacy{1}
\pmauthor{swiftset}{1337}
\pmtype{Definition}
\pmcomment{trigger rebuild}
\pmclassification{msc}{46C99}
\pmclassification{msc}{65T60}
\pmrelated{wavelet}
\pmrelated{Wavelet}

% this is the default PlanetMath preamble.  as your knowledge
% of TeX increases, you will probably want to edit this, but
% it should be fine as is for beginners.

% almost certainly you want these
\usepackage{amssymb}
\usepackage{amsmath}
\usepackage{amsfonts}

% used for TeXing text within eps files
%\usepackage{psfrag}
% need this for including graphics (\includegraphics)
%\usepackage{graphicx}
% for neatly defining theorems and propositions
%\usepackage{amsthm}
% making logically defined graphics
%%%\usepackage{xypic}

% there are many more packages, add them here as you need them

% define commands here
\begin{document}
\PMlinkescapeword{open}
\PMlinkescapeword{contain}
\PMlinkescapeword{inverse}

\paragraph{Definition}
An \emph{(orthonormal dyadic) wavelet set} on ${\mathbb R}$ is a subset $E \subset {\mathbb R}$ such that
\begin{enumerate}
\item $\chi_E \in L^2({\mathbb R})$ (since $\|\chi_E\| = \sqrt{m(E)}$, this implies $m(E) < \infty$).
\item $\frac{\chi_E}{\sqrt{m(E)}}$ is the Fourier transform of an orthonormal dyadic wavelet,
\end{enumerate}
where $\chi_E$ is the characteristic function of $E$, and $m(E)$ is the Lebesgue measure of $E$.

\paragraph{Characterization}
$E \subset {\mathbb R}$ is a wavelet set iff
\begin{enumerate}
\item $\{E + 2\pi n\}_{n\in {\mathbb Z}}$ is a measurable partition of $\mathbb R$; i.e. ${\mathbb R}\backslash \bigcup_{n\in \mathbb Z} \{ E + 2\pi n\}$ has measure zero, and $\bigcap_{n=i,j} \{E+2\pi n\}$ has measure zero if $i\neq j$. In short, $E$ is a $2\pi$-translation ``tiler'' of $\mathbb R$
\item $\{2^n E\}_{n\in \mathbb Z}$ is a $2$-dilation ``tiler'' of $\mathbb R$ (once again modulo sets of measure zero).
\end{enumerate}

\paragraph{Notes}
There are higher dimensional analogues to wavelet sets in $\mathbb R$, corresponding to wavelets in higher dimensions. Wavelet sets can be used to derive wavelets--- by creating a set $E$ satisfying the conditions given above, and using the inverse Fourier transform on $\chi_E$, you are guaranteed to recover a wavelet. A particularly interesting open question is: do all wavelets contain wavelet sets in their frequency support?
%%%%%
%%%%%
\end{document}
