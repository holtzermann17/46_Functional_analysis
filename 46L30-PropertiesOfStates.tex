\documentclass[12pt]{article}
\usepackage{pmmeta}
\pmcanonicalname{PropertiesOfStates}
\pmcreated{2013-03-22 17:45:24}
\pmmodified{2013-03-22 17:45:24}
\pmowner{asteroid}{17536}
\pmmodifier{asteroid}{17536}
\pmtitle{properties of states}
\pmrecord{5}{40209}
\pmprivacy{1}
\pmauthor{asteroid}{17536}
\pmtype{Theorem}
\pmcomment{trigger rebuild}
\pmclassification{msc}{46L30}
\pmclassification{msc}{46L05}

\endmetadata

% this is the default PlanetMath preamble.  as your knowledge
% of TeX increases, you will probably want to edit this, but
% it should be fine as is for beginners.

% almost certainly you want these
\usepackage{amssymb}
\usepackage{amsmath}
\usepackage{amsfonts}

% used for TeXing text within eps files
%\usepackage{psfrag}
% need this for including graphics (\includegraphics)
%\usepackage{graphicx}
% for neatly defining theorems and propositions
%\usepackage{amsthm}
% making logically defined graphics
%%%\usepackage{xypic}

% there are many more packages, add them here as you need them

% define commands here

\begin{document}
Let $\mathcal{A}$ be a \PMlinkname{$C^*$-algebra}{CAlgebra} and $x \in \mathcal{A}$. 

Let $S(\mathcal{A})$ and $P(\mathcal{A})$ denote the \PMlinkname{state}{State} space and the pure state space of $\mathcal{A}$, respectively.

\subsection{States}
The \PMlinkescapetext{state} space is sufficiently large to reveal many \PMlinkescapetext{properties} of elements of a $C^*$-algebra.

{\bf Theorem 1-} We have that
\begin{itemize}
\item $S(\mathcal{A})$ separates points, i.e. $x= 0$ if and only if $\phi(x) = 0$ for all $\phi \in S(\mathcal{A})$.
\item $x$ is \PMlinkname{self-adjoint}{InvolutaryRing} if and only if $\phi(x) \in \mathbb{R}$ for all $\phi \in S(\mathcal{A})$.
\item $x$ is positive if and only if $\phi(x) \geq 0$ for all $\phi \in S(\mathcal{A})$.
\item If $x$ is \PMlinkname{normal}{InvolutaryRing}, then $\phi(x) = \|x\|$ for some $\phi \in S(\mathcal{A})$.
\end{itemize}

\subsection{Pure states}

The pure state space is also sufficiently large to \PMlinkescapetext{satisfy} the \PMlinkescapetext{properties} of Theorem 1. Hence, we can replace $S(\mathcal{A})$ by $P(\mathcal{A})$, or by any other family of linear functionals $F$ such that $P(\mathcal{A}) \subset F \subset S(\mathcal{A})$, in the previous result.

{\bf Theorem 2 -} We have that
\begin{itemize}
\item $P(\mathcal{A})$ separates points, i.e. $x= 0$ if and only if $\phi(x) = 0$ for all $\phi \in P(\mathcal{A})$.
\item $x$ is \PMlinkescapetext{self-adjoint} if and only if $\phi(x) \in \mathbb{R}$ for all $\phi \in P(\mathcal{A})$.
\item $x$ is positive if and only if $\phi(x) \geq 0$ for all $\phi \in P(\mathcal{A})$.
\item If $x$ is \PMlinkescapetext{normal}, then $\phi(x) = \|x\|$ for some $\phi \in P(\mathcal{A})$.
\end{itemize}

{\bf \PMlinkescapetext{Proposition} -} Every multiplicative linear functional on $\mathcal{A}$ is a pure state.
%%%%%
%%%%%
\end{document}
