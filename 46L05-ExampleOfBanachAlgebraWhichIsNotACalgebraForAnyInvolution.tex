\documentclass[12pt]{article}
\usepackage{pmmeta}
\pmcanonicalname{ExampleOfBanachAlgebraWhichIsNotACalgebraForAnyInvolution}
\pmcreated{2013-03-22 17:25:54}
\pmmodified{2013-03-22 17:25:54}
\pmowner{asteroid}{17536}
\pmmodifier{asteroid}{17536}
\pmtitle{example of Banach algebra which is not a $C^*$-algebra for any involution}
\pmrecord{5}{39807}
\pmprivacy{1}
\pmauthor{asteroid}{17536}
\pmtype{Example}
\pmcomment{trigger rebuild}
\pmclassification{msc}{46L05}
\pmdefines{finite dimensional $C^*$-algebras are semi-simple}

% this is the default PlanetMath preamble.  as your knowledge
% of TeX increases, you will probably want to edit this, but
% it should be fine as is for beginners.

% almost certainly you want these
\usepackage{amssymb}
\usepackage{amsmath}
\usepackage{amsfonts}

% used for TeXing text within eps files
%\usepackage{psfrag}
% need this for including graphics (\includegraphics)
%\usepackage{graphicx}
% for neatly defining theorems and propositions
%\usepackage{amsthm}
% making logically defined graphics
%%%\usepackage{xypic}

% there are many more packages, add them here as you need them

% define commands here

\begin{document}
Consider the Banach algebra $\mathcal{A}= \left\{ 
\begin{bmatrix}
\lambda I_n & A \\
0 & \lambda I_n
\end{bmatrix} :\; \lambda \in \mathbb{C},\;\; A \in Mat_{n \times n}(\mathbb{C}) \right\}$ with the usual matrix operations and matrix norm, where $I_n$ denotes the identity matrix in $Mat_{n \times n}(\mathbb{C})$.

{\bf Claim -} $\mathcal{A}$ is not a \PMlinkname{$C^*$-algebra}{CAlgebra} for any involution $*$.

To prove the above claim we will give a \PMlinkescapetext{simple} proof of a more general fact about finite dimensional $C^*$-algebras, which clearly shows the \PMlinkescapetext{algebraic restrictions} for a Banach algebra to be a $C^*$-algebra for some involution.

{\bf Theorem -} Every finite dimensional $C^*$-algebra is semi-simple, i.e. its Jacobson radical is $\{0\}$.

{\bf Proof :} Let $\mathcal{B}$ be a finite dimensional $C^*$-algebra. Let $a$ be an element of $J(\mathcal{B})$, the Jacobson radical of $\mathcal{B}$.

$J(\mathcal{B})$ is an ideal of $\mathcal{B}$, so $a^*a \in J(\mathcal{B})$.

The Jacobson radical of a finite dimensional algebra is nilpotent, therefore there exists $n \in \mathbb{N}$ such that $(a^*a)^n=0$. Then, by the $C^*$ condition and the fact that $a^*a$ is selfadjoint,
\begin{displaymath}
0 = \|(a^*a)^{2^n}\| = \|a^*a\|^{2^n} = \|a\|^{2^{n+1}}
\end{displaymath}
so $a = 0$ and $J(\mathcal{B})$ is trivial. $\square$

We now prove the above claim.

{\bf Proof of the claim:} It is easy to see that 
$\left\{ 
\begin{bmatrix}
0 & A \\
0 & 0
\end{bmatrix} : \; A \in Mat_{n \times n}(\mathbb{C}) \right\}$ is the only maximal ideal of $\mathcal{A}$. Therefore the Jacobson radical of $\mathcal{A}$ is not trivial.

By the theorem we conclude that there is no involution $*$ that makes $\mathcal{A}$ into a $C^*$-algebra.$\square$

{\bf Remark -} It could happen that there were no involutions in $\mathcal{A}$ and so the above claim would be uninteresting. That's not the case here. For example, one can see that $[a_{i,j}] \longrightarrow [\overline{a}_{2n+1-j,2n+1-i}]$ defines an involution in $\mathcal{A}$ (this is just the \PMlinkescapetext{conjugate transpose} taken over the other diagonal of the matrix).
%%%%%
%%%%%
\end{document}
