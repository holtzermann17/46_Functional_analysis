\documentclass[12pt]{article}
\usepackage{pmmeta}
\pmcanonicalname{ResolventFunctionIsAnalytic}
\pmcreated{2013-03-22 17:29:36}
\pmmodified{2013-03-22 17:29:36}
\pmowner{asteroid}{17536}
\pmmodifier{asteroid}{17536}
\pmtitle{resolvent function is analytic}
\pmrecord{8}{39881}
\pmprivacy{1}
\pmauthor{asteroid}{17536}
\pmtype{Theorem}
\pmcomment{trigger rebuild}
\pmclassification{msc}{46H05}
\pmclassification{msc}{47A10}

% this is the default PlanetMath preamble.  as your knowledge
% of TeX increases, you will probably want to edit this, but
% it should be fine as is for beginners.

% almost certainly you want these
\usepackage{amssymb}
\usepackage{amsmath}
\usepackage{amsfonts}

% used for TeXing text within eps files
%\usepackage{psfrag}
% need this for including graphics (\includegraphics)
%\usepackage{graphicx}
% for neatly defining theorems and propositions
%\usepackage{amsthm}
% making logically defined graphics
%%%\usepackage{xypic}

% there are many more packages, add them here as you need them

% define commands here

\begin{document}
{\bf Theorem -} Let $\mathcal{A}$ be a complex Banach algebra with identity element $e$. Let $x \in \mathcal{A}$ and $\sigma(x)$ denote its spectrum. 

Then, the \PMlinkname{resolvent function}{ResolventMatrix} $R_x : \mathbb{C}-\sigma(x) \longrightarrow \mathcal{A}$ defined by $R_x(\lambda) = (x-\lambda e)^{-1}$ is \PMlinkname{analytic}{BanachSpaceValuedAnalyticFunctions}. 

Moreover, for each $\lambda_0 \in \mathbb{C} - \sigma(x)$ it has the power series \PMlinkescapetext{representation}
\begin{align}
R_x(\lambda) = \sum_{n=0}^{\infty} R_x(\lambda_0)^{n+1}(\lambda -\lambda_0)^n
\end{align}
where the series converges absolutely for each $\lambda$ in the open disk centered in $\lambda_0$ given by
\begin{align}
|\lambda - \lambda_0| < \frac{1}{\|R_x(\lambda_0)\|}
\end{align}

{\bf Proof :} Analyticity is defined for functions whose domain is open. 

Thus, we start by proving that $\mathbb{C} - \sigma(x)$ is an open set in $\mathbb{C}$. To do so it is enough to prove that for every $\lambda_0 \in \mathbb{C} - \sigma(x)$ the open disk defined by (2) above is contained in $\mathbb{C} - \sigma(x)$.

Let $\lambda_0 \in \mathbb{C} - \sigma(x)$ and $\lambda$ be such that
\begin{displaymath}
|\lambda - \lambda_0| < \frac{1}{\|R_x(\lambda_0)\|}
\end{displaymath}

Then $\|(\lambda - \lambda_0)R_x(\lambda_0)\| < 1$ and by the \PMlinkname{Neumann series}{NeumannSeriesInBanachAlgebras} $e - (\lambda - \lambda_0)R_x(\lambda_0)$ is invertible.

Since $\lambda_0 \notin \sigma(x)$ it follows that $(x-\lambda_0 e)$ is invertible.

Hence, from the equality
\begin{align}
x-\lambda e = x- \lambda_0 e - (\lambda - \lambda_0)e = (x-\lambda_0 e)\cdot [e - (\lambda - \lambda_0)R_x(\lambda_0)]
\end{align}
we conclude that $x - \lambda e$ is also invertible, i.e. $\lambda \in \mathbb{C} - \sigma(x)$. Thus $\mathbb{C} - \sigma(x)$ is open.

The above proof also pointed out that for every $\lambda_0 \in \mathbb{C}$, $R_x$ is defined in the open disk of radius $\displaystyle \frac{1}{\|R_x(\lambda_0)\|}$ centered in $\lambda_0$.

We now prove the analyticity of the \PMlinkescapetext{resolvent function}.

Taking inverses on the equality (3) above one obtains
\begin{displaymath}
R_x(\lambda) = (e-(\lambda-\lambda_0)R_x(\lambda_0))^{-1} \cdot R_x(\lambda_0)
\end{displaymath}

Again, by the \PMlinkname{Neumann series}{NeumannSeriesInBanachAlgebras}, one obtains
\begin{displaymath}
R_x(\lambda) = \left[ \sum_{n=0}^{\infty} R_x(\lambda_0)^{n}(\lambda -\lambda_0)^n \right]\cdot R_x(\lambda_0) =
\sum_{n=0}^{\infty} R_x(\lambda_0)^{n+1}(\lambda -\lambda_0)^n  \;\;\;\;\square
\end{displaymath}
%%%%%
%%%%%
\end{document}
