\documentclass[12pt]{article}
\usepackage{pmmeta}
\pmcanonicalname{WeakTopologyOfTheSpaceOfRadonMeasures}
\pmcreated{2013-03-22 15:22:58}
\pmmodified{2013-03-22 15:22:58}
\pmowner{stevecheng}{10074}
\pmmodifier{stevecheng}{10074}
\pmtitle{weak-* topology of the space of Radon measures}
\pmrecord{4}{37212}
\pmprivacy{1}
\pmauthor{stevecheng}{10074}
\pmtype{Example}
\pmcomment{trigger rebuild}
\pmclassification{msc}{46A03}
\pmclassification{msc}{28A33}
\pmrelated{LocallyCompactHausdorffSpace}

\endmetadata

% this is the default PlanetMath preamble.  as your knowledge
% of TeX increases, you will probably want to edit this, but
% it should be fine as is for beginners.

% almost certainly you want these
\usepackage{amssymb}
\usepackage{amsmath}
\usepackage{amsfonts}

% used for TeXing text within eps files
%\usepackage{psfrag}
% need this for including graphics (\includegraphics)
%\usepackage{graphicx}
% for neatly defining theorems and propositions
%\usepackage{amsthm}
% making logically defined graphics
%%%\usepackage{xypic}

% there are many more packages, add them here as you need them

% define commands here
\begin{document}
Let $X$ be a locally compact Hausdorff space.
Let $M(X)$ denote the space of complex Radon measures on $X$, and
$C_0(X)^*$ denote the dual of the $C_0(X)$, the complex-valued continuous functions on $X$
vanishing at infinity, equipped with the uniform norm.
By the Riesz Representation Theorem, $M(X)$ is isometric to $C_0(X)^*$,
The isometry maps a measure $\mu$ into the linear functional $I_\mu(f) = \int_X f \, d\mu$.

The weak-* topology (also called the vague topology) on $C_0(X)^*$,
is simply the topology of pointwise convergence of $I_\mu$: 
$I_{\mu_\alpha} \to I_{\mu}$ if and only if
$I_{\mu_\alpha}(f) \to I_{\mu}(f)$ for each $f \in C_0(X)$.

The corresponding topology on $M(X)$ induced by the isometry from $C_0(X)^*$ is also called
the weak-* or vague topology on $M(X)$.  Thus one may talk about ``weak convergence'' of measures 
$\mu_n \to \mu$.  One of the most important applications of this notion is in probability theory:
for example, the central limit theorem is essentially the statement that
if $\mu_n$ are the distributions for certain sums of independent random variables.
then $\mu_n$ converge weakly to a normal distribution,
i.e. the distribution $\mu_n$ is ``approximately normal'' for large $n$.

\begin{thebibliography}{9}
\bibitem{folland}
G.B. Folland, \emph{Real Analysis: Modern Techniques and Their Applications}, 2nd ed, John Wiley \& Sons, Inc., 1999.
\end{thebibliography}
%%%%%
%%%%%
\end{document}
