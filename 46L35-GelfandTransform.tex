\documentclass[12pt]{article}
\usepackage{pmmeta}
\pmcanonicalname{GelfandTransform}
\pmcreated{2013-03-22 17:22:39}
\pmmodified{2013-03-22 17:22:39}
\pmowner{asteroid}{17536}
\pmmodifier{asteroid}{17536}
\pmtitle{Gelfand transform}
\pmrecord{26}{39743}
\pmprivacy{1}
\pmauthor{asteroid}{17536}
\pmtype{Definition}
\pmcomment{trigger rebuild}
\pmclassification{msc}{46L35}
\pmclassification{msc}{46L05}
\pmclassification{msc}{46J40}
\pmclassification{msc}{46J05}
\pmclassification{msc}{46H05}
%\pmkeywords{Gelfand transform}
%\pmkeywords{C*-algebra}
%\pmkeywords{C*- algebra representations}
%\pmkeywords{CompactQuantumGroup}
\pmrelated{MultiplicativeLinearFunctional}
\pmrelated{NoncommutativeTopology}
\pmrelated{CAlgebra3}
\pmrelated{CAlgebra}
\pmrelated{CompactQuantumGroup}
\pmdefines{classification of commutative $C^*$-algebras}
\pmdefines{commutative $C^*$-algebras classification}
\pmdefines{Gelfand-Naimark theorem}

% this is the default PlanetMath preamble.  as your knowledge
% of TeX increases, you will probably want to edit this, but
% it should be fine as is for beginners.

% almost certainly you want these
\usepackage{amssymb}
\usepackage{amsmath}
\usepackage{amsfonts}

% used for TeXing text within eps files
%\usepackage{psfrag}
% need this for including graphics (\includegraphics)
%\usepackage{graphicx}
% for neatly defining theorems and propositions
%\usepackage{amsthm}
% making logically defined graphics
%%%\usepackage{xypic}

% there are many more packages, add them here as you need them

% define commands here

\begin{document}
\subsection*{The Gelfand Transform} Let $\mathcal{A}$ be a Banach algebra over $\mathbb{C}$.
 Let $\bigtriangleup$ be the space of all
 multiplicative linear functionals in $\mathcal{A}$, endowed with the weak-* topology. Let
 $C(\bigtriangleup)$ denote the algebra of complex valued continuous functions in $\bigtriangleup$.

The {\bf Gelfand transform} is the mapping
\begin{center}

$\widehat{}\;\;:\mathcal{A} \longrightarrow C(\bigtriangleup)$

$x \longmapsto \widehat{x}$
\end{center}

where $\widehat{x} \in C(\bigtriangleup)$ is defined by
 $\;\;\widehat{x} (\phi) := \phi(x), \;\;\;\forall \phi \in \bigtriangleup$

The Gelfand transform is a continuous homomorphism from $\mathcal{A}$ to $C(\bigtriangleup)$.

{\bf Theorem -} Let $C_{0}(\bigtriangleup)$ denote the algebra of complex valued continuous functions in $\bigtriangleup$, that vanish at infinity.
 The image of the Gelfand transform is contained in $C_{0}(\bigtriangleup)$.


The Gelfand transform is a very useful tool in the study of commutative Banach algebras and, particularly,
 commutative \PMlinkname{$C^*$-algebras}{CAlgebra}.

\subsection*{Classification of commutative $C^*$-algebras: Gelfand-Naimark theorems}

The following results are called the Gelfand-Naimark theorems. They classify all commutative $C^*$-algebras and all commutative $C^*$-algebras with identity element.

{\bf Theorem 1 -} Let $\mathcal{A}$ be a $C^*$-algebra over $\mathbb{C}$. Then $\mathcal{A}$ is
 *-isomorphic to $C_{0}(X)$ for some locally compact Hausdorff space $X$. Moreover, the Gelfand transform is a
 *-isomorphism between $\mathcal{A}$ and $C_{0}(\bigtriangleup)$.

{\bf Theorem 2 -} Let $\mathcal{A}$ be a unital $C^*$-algebra over $\mathbb{C}$. Then $\mathcal{A}$ is
 *-isomorphic to $C(X)$ for some compact Hausdorff space $X$. Moreover, the Gelfand transform is a
 *-isomorphism between $\mathcal{A}$ and $C(\bigtriangleup)$.

The above theorems can be substantially improved. In fact, there is an \PMlinkname{equivalence}{EquivalenceOfCategories} between the category of commutative $C^*$-algebras and the category of locally compact Hausdorff spaces. For more \PMlinkescapetext{information} and details about this, see the entry about the general \PMlinkname{Gelfand-Naimark theorem}{GelfandNaimarkTheorem}.
%%%%%
%%%%%
\end{document}
