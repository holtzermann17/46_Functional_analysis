\documentclass[12pt]{article}
\usepackage{pmmeta}
\pmcanonicalname{TopologicallyNilpotent}
\pmcreated{2013-03-22 16:12:04}
\pmmodified{2013-03-22 16:12:04}
\pmowner{CWoo}{3771}
\pmmodifier{CWoo}{3771}
\pmtitle{topologically nilpotent}
\pmrecord{6}{38295}
\pmprivacy{1}
\pmauthor{CWoo}{3771}
\pmtype{Definition}
\pmcomment{trigger rebuild}
\pmclassification{msc}{46H05}
\pmsynonym{quasinilpotent}{TopologicallyNilpotent}

\endmetadata

\usepackage{amssymb,amscd}
\usepackage{amsmath}
\usepackage{amsfonts}

% used for TeXing text within eps files
%\usepackage{psfrag}
% need this for including graphics (\includegraphics)
%\usepackage{graphicx}
% for neatly defining theorems and propositions
%\usepackage{amsthm}
% making logically defined graphics
%%\usepackage{xypic}
\usepackage{pst-plot}
\usepackage{psfrag}

% define commands here

\begin{document}
An element $a$ in a normed ring $A$ is said to be \emph{topologically nilpotent} if $$\lim_{n\to\infty} \|a^n\|^{\frac{1}{n}}=0.$$

Topologically nilpotent elements are also called \emph{quasinilpotent}.

\textbf{Remarks}. 
\begin{itemize}
\item Any nilpotent element is topologically nilpotent.
\item If $a$ and $b$ are topologically nilpotent and $ab=ba$, then $ab$ is topologically nilpotent.
\item When $A$ is a unital Banach algebra, an element $a \in A$ is topologically nilpotent iff its spectrum $\sigma(a)$ equals $\{0\}$.
\end{itemize}
%%%%%
%%%%%
\end{document}
