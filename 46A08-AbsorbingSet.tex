\documentclass[12pt]{article}
\usepackage{pmmeta}
\pmcanonicalname{AbsorbingSet}
\pmcreated{2013-03-22 15:26:24}
\pmmodified{2013-03-22 15:26:24}
\pmowner{CWoo}{3771}
\pmmodifier{CWoo}{3771}
\pmtitle{absorbing set}
\pmrecord{10}{37287}
\pmprivacy{1}
\pmauthor{CWoo}{3771}
\pmtype{Definition}
\pmcomment{trigger rebuild}
\pmclassification{msc}{46A08}
\pmclassification{msc}{15A03}
\pmrelated{BalancedSet}
\pmrelated{AbsorbingElement}
\pmdefines{absorbing}
\pmdefines{absorb}
\pmdefines{radial}

\endmetadata

\usepackage{amssymb,amscd}
\usepackage{amsmath}
\usepackage{amsfonts}

% used for TeXing text within eps files
%\usepackage{psfrag}
% need this for including graphics (\includegraphics)
%\usepackage{graphicx}
% for neatly defining theorems and propositions
%\usepackage{amsthm}
% making logically defined graphics
%%%\usepackage{xypic}

% define commands here

\newcommand{\abs}[1]{\vert{#1}\vert}
\begin{document}
\PMlinkescapeword{symmetric}

Let $V$ be a vector space over a field $F$ equipped with a
non-discrete valuation $\abs{\cdot}:F\to\mathbb{R}$.  Let $A,B$ be
two subsets of $V$.  Then $A$ is said to \emph{absorb} $B$ if there
is a non-negative real number $r$ such that, for all $\lambda\in F$
with $\abs{\lambda}\geq r$, $B\subseteq\lambda A$.  $A$ is said to
be an \emph{absorbing set}, or a \emph{radial subset} of $V$ if $A$
absorbs all finite subsets of $V$.

Equivalently, $A$ is absorbing if for any $x\in V$, there is a
non-negative real number $r$ such that $x\in\lambda A$ for all
$\lambda\in F$ with $\abs{\lambda}\geq r$.  If a finite subset $B$
of $V$ consists of $x_1,\ldots,x_n$, then corresponding to each
$x_i$, there is an $r_i\geq 0$ such that $x_i\in\lambda A$ such that
$\vert\lambda\mid\geq r_i$, $\forall\lambda\in F$.  So
$x_i\in\lambda A$ with $\abs{\lambda}\geq r$ if we take
$r=\max\lbrace r_1,\ldots,r_n\rbrace$.  So $A$ absorbs $B$.

\textbf{Example}. If $V=\mathbb{R}^n$ and $F=\mathbb{R}$, then any
set containing an open ball centered at $0$ is absorbing. This
implies that an absorbing set does not have to be connected, convex.

A closely related concept is that of a \emph{circled set}, or a \emph{balanced set}. Let $V$ and $F$ be defined as above.  A subset $C$ of $V$ is said to be
\emph{circled}, or \emph{balanced}, if $\lambda C\subseteq C$ for all $\abs{\lambda}\leq 1$.  Clearly, $C$ absorbs itself ($C\subseteq\lambda^{-1}C$,
$\abs{\lambda^{-1}}\geq 1$), and $0\in C$. $C$ is also symmetric
($-C=C$), for $-C\subseteq C$ and $C=-(-C)\subseteq -C$. As an
example of a circled set that is neither absorbing nor convex,
consider $V=\mathbb{R}^2$ and $F=\mathbb{R}$, and $C$ the union of
$x$ and $y$ axes.  For an example of an absorbing set that is not
circled, take the union of a unit disk and an annulus centered at 0
that is large enough so it is disjoint from the disk.
%%%%%
%%%%%
\end{document}
