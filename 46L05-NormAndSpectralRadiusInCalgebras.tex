\documentclass[12pt]{article}
\usepackage{pmmeta}
\pmcanonicalname{NormAndSpectralRadiusInCalgebras}
\pmcreated{2013-03-22 17:38:41}
\pmmodified{2013-03-22 17:38:41}
\pmowner{asteroid}{17536}
\pmmodifier{asteroid}{17536}
\pmtitle{norm and spectral radius in $C^*$-algebras}
\pmrecord{7}{40067}
\pmprivacy{1}
\pmauthor{asteroid}{17536}
\pmtype{Theorem}
\pmcomment{trigger rebuild}
\pmclassification{msc}{46L05}
\pmrelated{HomomorphismsOfCAlgebrasAreContinuous}
\pmrelated{CAlgebra}

% this is the default PlanetMath preamble.  as your knowledge
% of TeX increases, you will probably want to edit this, but
% it should be fine as is for beginners.

% almost certainly you want these
\usepackage{amssymb}
\usepackage{amsmath}
\usepackage{amsfonts}

% used for TeXing text within eps files
%\usepackage{psfrag}
% need this for including graphics (\includegraphics)
%\usepackage{graphicx}
% for neatly defining theorems and propositions
%\usepackage{amsthm}
% making logically defined graphics
%%%\usepackage{xypic}

% there are many more packages, add them here as you need them

% define commands here

\begin{document}
Let $\mathcal{A}$ be a \PMlinkname{$C^*$-algebra}{CAlgebra}. Let $R_{\sigma}(a)$ denote the spectral radius of the element $a \in \mathcal{A}$.

{\bf Theorem -} For every $a \in \mathcal{A}$ we have that $\|a\| = \sqrt{R_{\sigma}(a^*a)}$.

This result shows that the norm in a $C^*$-algebra has a purely \PMlinkescapetext{algebraic} nature. Moreover, the norm in a $C^*$-algebra is unique (in the sense that there is no other norm for which the algebra is a $C^*$-algebra).
%%%%%
%%%%%
\end{document}
