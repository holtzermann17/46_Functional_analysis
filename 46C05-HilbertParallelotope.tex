\documentclass[12pt]{article}
\usepackage{pmmeta}
\pmcanonicalname{HilbertParallelotope}
\pmcreated{2013-03-22 14:38:32}
\pmmodified{2013-03-22 14:38:32}
\pmowner{rspuzio}{6075}
\pmmodifier{rspuzio}{6075}
\pmtitle{Hilbert parallelotope}
\pmrecord{6}{36229}
\pmprivacy{1}
\pmauthor{rspuzio}{6075}
\pmtype{Definition}
\pmcomment{trigger rebuild}
\pmclassification{msc}{46C05}
\pmsynonym{Hilbert cube}{HilbertParallelotope}

% this is the default PlanetMath preamble.  as your knowledge
% of TeX increases, you will probably want to edit this, but
% it should be fine as is for beginners.

% almost certainly you want these
\usepackage{amssymb}
\usepackage{amsmath}
\usepackage{amsfonts}

% used for TeXing text within eps files
%\usepackage{psfrag}
% need this for including graphics (\includegraphics)
%\usepackage{graphicx}
% for neatly defining theorems and propositions
%\usepackage{amsthm}
% making logically defined graphics
%%%\usepackage{xypic}

% there are many more packages, add them here as you need them

% define commands here
\begin{document}
The \emph{Hilbert parallelotope} $I^\omega$ is a closed subset of the Hilbert space $\mathbb{R} \ell^2$ (The symbol '$\mathbb{R}$' has been prefixed to indicate that the field of scalars is $\mathbb{R}$.) defined as
 $$I^\omega = \{(a_0, a_1, a_2, \ldots) \mid 0 \le a_i \le 1/(i+1) \}$$

As a topological space, $I^\omega$ is homeomorphic to the product of a countably infinite number of copies of the closed interval $[0,1]$.  By Tychonoff's theorem, this product is compact, so the Hilbert parallelotope is a compact subset of Hilbert space.  This fact also explains the notation $I^\omega$.

The Hilbert parallelotope enjoys a remarkable universality property --- every second countable metric space is homeomorphic to a subset of the Hilbert parallelotope.  Since second countability is hereditary, the converse is also true --- every subset of the Hilbert parallelotope is a second countable metric space.
%%%%%
%%%%%
\end{document}
