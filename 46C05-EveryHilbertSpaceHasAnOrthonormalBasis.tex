\documentclass[12pt]{article}
\usepackage{pmmeta}
\pmcanonicalname{EveryHilbertSpaceHasAnOrthonormalBasis}
\pmcreated{2013-03-22 17:56:05}
\pmmodified{2013-03-22 17:56:05}
\pmowner{asteroid}{17536}
\pmmodifier{asteroid}{17536}
\pmtitle{every Hilbert space has an orthonormal basis}
\pmrecord{4}{40431}
\pmprivacy{1}
\pmauthor{asteroid}{17536}
\pmtype{Theorem}
\pmcomment{trigger rebuild}
\pmclassification{msc}{46C05}

% this is the default PlanetMath preamble.  as your knowledge
% of TeX increases, you will probably want to edit this, but
% it should be fine as is for beginners.

% almost certainly you want these
\usepackage{amssymb}
\usepackage{amsmath}
\usepackage{amsfonts}

% used for TeXing text within eps files
%\usepackage{psfrag}
% need this for including graphics (\includegraphics)
%\usepackage{graphicx}
% for neatly defining theorems and propositions
%\usepackage{amsthm}
% making logically defined graphics
%%%\usepackage{xypic}

% there are many more packages, add them here as you need them

% define commands here

\begin{document}
{\bf Theorem -} Every Hilbert space $H \neq \{0\}$ has an orthonormal basis.

$\,$

{\bf \emph{Proof :}} As could be expected, the proof makes use of Zorn's Lemma. Let $\mathcal{O}$ be the set of all orthonormal sets of $H$. It is clear that $\mathcal{O}$ is non-empty since the set $\{x\}$ is in $\mathcal{O}$, where $x$ is an element of $H$ such that $\|x\| = 1$.

The elements of $\mathcal{O}$ can be ordered by inclusion, and each chain $\mathcal{C}$ in $\mathcal{O}$ has an upper bound, given by the union of all elements of $\mathcal{C}$. Thus, Zorn's Lemma assures the existence of a maximal element $B$ in $\mathcal{O}$. We claim that $B$ is an orthonormal basis of $H$.

It is clear that $B$ is an orthonormal set, as it belongs to $\mathcal{O}$. It remains to see that the linear span of $B$ is dense in $H$.

Let $\overline{\mathrm{span}\,B}$ denote the closure of the span of $B$. Suppose $\overline{\mathrm{span}\,B} \neq H$. By the orthogonal decomposition theorem we know that
\begin{displaymath}
H = \overline{\mathrm{span}\,B} \oplus (\overline{\mathrm{span}\,B})^{\perp}  
\end{displaymath}
Thus, we conclude that $(\overline{\mathrm{span}\,B})^{\perp} \neq \{0\}$, i.e. there are elements which are \PMlinkname{orthogonal}{OrthogonalVectors} to $\overline{\mathrm{span}\,B}$. This contradicts the maximality of $B$ since, by picking an element $y \in (\overline{\mathrm{span}\,B})^{\perp}$ with $\|y\| = 1$,  $B \cup \{y\}$ would belong belong to $\mathcal{O}$ and would be greater than $B$.

Hence, $\overline{\mathrm{span}\,B} = H$, and this finishes the proof. $\square$
%%%%%
%%%%%
\end{document}
