\documentclass[12pt]{article}
\usepackage{pmmeta}
\pmcanonicalname{ProofOfBanachSteinhausTheorem}
\pmcreated{2013-03-22 14:48:41}
\pmmodified{2013-03-22 14:48:41}
\pmowner{Koro}{127}
\pmmodifier{Koro}{127}
\pmtitle{proof of Banach-Steinhaus theorem}
\pmrecord{6}{36470}
\pmprivacy{1}
\pmauthor{Koro}{127}
\pmtype{Proof}
\pmcomment{trigger rebuild}
\pmclassification{msc}{46B99}

\endmetadata

% this is the default PlanetMath preamble.  as your knowledge
% of TeX increases, you will probably want to edit this, but
% it should be fine as is for beginners.

% almost certainly you want these
\usepackage{amssymb}
\usepackage{amsmath}
\usepackage{amsfonts}
\usepackage{mathrsfs}

% used for TeXing text within eps files
%\usepackage{psfrag}
% need this for including graphics (\includegraphics)
%\usepackage{graphicx}
% for neatly defining theorems and propositions
%\usepackage{amsthm}
% making logically defined graphics
%%%\usepackage{xypic}

% there are many more packages, add them here as you need them

% define commands here
\newcommand{\C}{\mathbb{C}}
\newcommand{\R}{\mathbb{R}}
\newcommand{\N}{\mathbb{N}}
\newcommand{\Z}{\mathbb{Z}}
\newcommand{\Per}{\operatorname{Per}}
\begin{document}
Let
$$E_n = \{x\in X: \|T(x)\|\leq n\textnormal{ for all }T\in \mathcal{F}\}.$$
From the hypothesis, we have that
$$\bigcup_{n=1}^\infty E_n = X.$$
Also, each $E_n$ is closed, since it can be written as
$$E_n = \bigcap_{T\in\mathcal{F}}{T^{-1}(B(0,n))},$$
where $B(0,n)$ is the closed ball centered at $0$ with radius $n$ in $Y$,
and each of the sets in the intersection is closed due to the continuity of the operators.
Now since $X$ is a Banach space, Baire's category theorem 
implies that there exists $n$ such that $E_n$ has
nonempty interior. So there is $x_0\in E_n$ and $r>0$ such 
that $B(x_0,r)\subset E_n$. Thus if $\|x\|\leq r$, we have
$$\|T(x)\|-\|T(x_0)\|\leq \|T(x_0)+T(x)\|=\|T(x_0+x)\|\leq n$$
for each $T\in \mathcal{F}$, and so
$$\|T(x)\|\leq n+\|T(x_0)\|$$
so if $\|x\|\leq 1$, we have 
$$\|T(x)\|= \frac{1}{r}\|T(rx)\| \leq \frac{1}{r}\left(n+\|T(x_0)\|\right) = c,$$
and this means that
$$\|T\| = \sup\{\|Tx\|: \|x\|\leq 1\} \leq c$$
for all $T\in\mathcal{F}$.
%%%%%
%%%%%
\end{document}
