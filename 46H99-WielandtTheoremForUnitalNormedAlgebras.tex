\documentclass[12pt]{article}
\usepackage{pmmeta}
\pmcanonicalname{WielandtTheoremForUnitalNormedAlgebras}
\pmcreated{2013-03-22 19:01:22}
\pmmodified{2013-03-22 19:01:22}
\pmowner{karstenb}{16623}
\pmmodifier{karstenb}{16623}
\pmtitle{Wielandt theorem for unital normed algebras}
\pmrecord{6}{41894}
\pmprivacy{1}
\pmauthor{karstenb}{16623}
\pmtype{Theorem}
\pmcomment{trigger rebuild}
\pmclassification{msc}{46H99}
%\pmkeywords{commutator}
%\pmkeywords{hilbert space}
%\pmkeywords{uncertainty principle}

\usepackage{amssymb}
\usepackage{amsmath}
\usepackage{amsfonts}
\usepackage{amsthm}
\usepackage{mathrsfs}
\usepackage[sort&compress]{natbib}

%\usepackage{psfrag}
%\usepackage{graphicx}
%%%\usepackage{xypic}

%theorems
\theoremstyle{definition}
\newtheorem{Def}{Definition}

\theoremstyle{plain}
\newtheorem{Lem}{Theorem}
\newtheorem{Lem2}{Lemma}
\newtheorem{Cor}{Corollary}
\newtheorem{Rem}{Remark}

\begin{document}
\textbf{Theorem.} (Wielandt (1949)) Let $A$ be a normed unital algebra (with unit $e$). If $x, y \in A$ then $xy - yx \not= e$.

\begin{proof}
Assume there are $x,y \in A$ such that $xy - yx = e$. Then for all $n \in \mathbb{N}$ we have 
\begin{align*}
x^n y - y x^n &= n x^{n-1} \not= 0
\end{align*}

We prove this by induction over $n \in \mathbb{N}$. It holds for $n = 1$ by assumption. Assume it is valid for $n \in \mathbb{N}$. Then $x^n \not= 0$ and
\begin{align*}
x^{n+1} y - y x^{n+1} &= x^n (x y - y x) + (x^n y - y x^n) x \\
&= x^n e + n x^{n-1} x = x^n e + n x^n = (n+e) x^n
\end{align*}

From this identity it follows that
\begin{align*}
n \|x^{n-1}\| &= \|x^n y - y x^n\| \leq 2 \|x^n\| \|y\| \leq 2 \|x^{n-1}\| \|x\| \|y\| \\
\end{align*}

It follows that $n \leq 2 ||x|| ||y||$ for all $n \in \mathbb{N}$ which is impossible. 
\end{proof}


\textbf{Corollary.} The identity operator on a Hilbert space $\mathcal{H}$ cannot be expressed as a commutator of two bounded linear operators in $\mathcal{L}(\mathcal{H})$.


\textbf{Remark.} The above can be understood as a version of the uncertainty principle in one dimension. Let $H = L^2(\mathbb{R})$. Let $q \colon H \to H$ be $q(f)(x) := xf(x)$ with $D(q) = \{f \in L^2(\mathbb{R}) : x \mapsto x f(x) \in L^2(\mathbb{R})\}$, the coordinate operator and $p \colon H \to H, p(f)(x) = -i f'(x)$ the momentum operator with $D(p) := \{f \in L^2(\mathbb{R}) : f \ \text{absolutely \ continuous}, f' \in L^2(\mathbb{R})\}$. It follows that 
\begin{align*}
pq - qp &= -i \mathrm{id}_D \ \text{on} \ D = D(q) \cap D(p) 
\end{align*}

According to the corollary $D = L^2(\mathbb{R})$ can never be the case.

%%%%%
%%%%%
\end{document}
