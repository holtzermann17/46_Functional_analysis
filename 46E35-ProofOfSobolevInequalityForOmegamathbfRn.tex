\documentclass[12pt]{article}
\usepackage{pmmeta}
\pmcanonicalname{ProofOfSobolevInequalityForOmegamathbfRn}
\pmcreated{2013-03-22 15:05:22}
\pmmodified{2013-03-22 15:05:22}
\pmowner{vanschaf}{8061}
\pmmodifier{vanschaf}{8061}
\pmtitle{proof of Sobolev inequality for $\Omega=\mathbf{R}^n$}
\pmrecord{14}{36816}
\pmprivacy{1}
\pmauthor{vanschaf}{8061}
\pmtype{Proof}
\pmcomment{trigger rebuild}
\pmclassification{msc}{46E35}

% this is the default PlanetMath preamble.  as your knowledge
% of TeX increases, you will probably want to edit this, but
% it should be fine as is for beginners.

% almost certainly you want these
\usepackage{amssymb}
\usepackage{amsmath}
\usepackage{amsfonts}

% used for TeXing text within eps files
%\usepackage{psfrag}
% need this for including graphics (\includegraphics)
%\usepackage{graphicx}
% for neatly defining theorems and propositions
%\usepackage{amsthm}
% making logically defined graphics
%%%\usepackage{xypic}

% there are many more packages, add them here as you need them

% define commands here
\begin{document}
\paragraph{Step 1: $u$ is smooth and $p=1$}
First suppose $u$ is a compactly supported smooth function, and let $(e_k)_{1 \le k \le n}$ denote a basis of $\mathbf{R}^n$. For every $1 \le k \le n$,
\[
  u(x)=\int_{-\infty}^0 \frac{\partial u}{\partial x_k}(x+se_k)\,ds.
\]
Therefore,
\[
  \lvert u(x)\rvert \le S_k(x):=\int_{\mathbf{R}} \bigl\lvert\frac{\partial u}{\partial x_k}(x_1,\dots,x_{k-1},s,x_{k+1},\dots,x_n)\bigr\rvert\,ds.
\]
Note that $S_k$ does not depend on $x_k$.
One also has
\[
  \lvert u(x)\rvert^{n/(n-1)}\le \prod_{k=1}^n \lvert S_k(x)\rvert^{1/(n-1)}.
\]
The integration of this inequality yields,
\[
  \int_{\mathbf{R}^n} \vert u(x)\vert^{n/(n-1)}\,dx\le \int_{\mathbf{R}^n}\prod_{k=1}^n \lvert S_k(x)\rvert^{1/(n-1)}\,dx.
\]
Since $S_1$ does not depend on $x_k$, we can apply the generalized H\"older inequality with $n-1$ for the integration with respect to $x_1$ in order to obtain:
\[
  \int_{\mathbf{R}^n} \lvert u(x)\rvert^{n/(n-1)}\,dx\le \int_{\mathbf{R}^{n-1}} S_1(x)\prod_{k=2}^n \Bigl(\int_{\mathbf{R}} S_k(x)\,dx_1\Bigr)^{1/(n-1)}\,dx_1\dots dx_n.
\]
The repetition of this process for the variables $x_2,\dots,x_n$ gives
\[
  \int_{\mathbf{R}^n} \lvert u(x)\rvert^{n/(n-1)}\,dx\le \prod_{k=1}^n \Bigl(\int_{\mathbf{R}^n} \bigl\lvert \frac{\partial u}{\partial x_k}\bigr\rvert\,dx\Bigr)^{1/(n-1)}.
\]
By the arithmetic-geometric means inequality, one obtains
\[
  \int_{\mathbf{R}^n} \lvert u(x)\rvert^{n/(n-1)}\,dx\le n^ {-n/(n-1)}\Bigl(\sum_{k=1}^n \Bigl(\int_{\mathbf{R}^n} \bigl\lvert \frac{\partial u}{\partial x_k}\bigr\rvert\,dx\Bigr)\Bigr)^{n/(n-1)}.
\]
One finally concludes
\[
  \lVert u \rVert_{L^{n/(n-1)}} \le n^{1/2-n/(n-1)} \lVert \nabla u \rVert_{L^{n/(n-1)}}.
\]

\paragraph{Step 2: general $u$ and $p=1$}
In general if $u \in W^{1,1}(\mathbf{R}^n)$. It can be approximated by a sequence of compactly supported smooth functions $(u_m)$. By step 1, one has
\[
  \lVert u_m-u_\ell \rVert_{L^{n/(n-1)}} \le n^{1/2-n/(n-1)} \lVert \nabla u_m-\nabla u_\ell \rVert_{L^{1}}.
\]
therefore $(u_m)$ is a Cauchy sequence in $L^{n/(n-1)}(\mathbf{R}^n)$. Since it converges to $u$ in $L^{1}(\mathbf{R}^n)$, the limit of $(u_m)$ is $u$ in $L^{n/(n-1)}(\mathbf{R}^n)$ and one has
\[
  \lVert u \rVert_{L^{n/(n-1)}} \le n^{1/2-n/(n-1)} \lVert \nabla u \rVert_{L^{n/(n-1)}}.  
\]

\paragraph{Step 3: $1<p <n$ and $u$ is smooth}
Suppose $1 <p <n $ and $u$ is a smooth compactly supported function. 
Let 
\[
  r=\frac{p(n-1)}{n-p}
\]
and
\[
  v=\lvert u \rvert^r.
\]
Since $u$ is smooth, $v \in W^{1,1}$ (It is however not necessarily smooth), and its weak derivative is
\[
  \nabla v=r u \lvert u \rvert^{r-2} \nabla u.
\]
One has, by the H\"older inequality,
\[
  \rVert \nabla v \lVert_{L^1(\mathbf{R}^N)}\le r \rVert \lvert u\rvert^r \lVert_{L^{p/p-1}(\mathbf{R}^N)}\rVert \nabla u \lVert_{L^{p}(\mathbf{R}^N)}
= r \rVert u \lVert_{L^{np/(n-p)}(\mathbf{R}^N)}^{r-1}\rVert \nabla u \lVert_{L^{p}(\mathbf{R}^N)}
\]
Therefore, the Sobolev inequality yields
\[
  \rVert u \lVert_{L^{np/(n-p)}(\mathbf{R}^N)}^r
= \rVert v \lVert_{L^{n/(n-1)}(\mathbf{R}^N)}
\le r n^{1/2-n/(n-1)}\rVert u \lVert_{L^{np/(n-p)}(\mathbf{R}^N)}^{r-1}\rVert \nabla u \lVert_{L^{p}(\mathbf{R}^N)}.
\]
This yields
\[
  \rVert u \lVert_{L^{np/(n-p)}(\mathbf{R}^N)}
\le n^{1/2-n/(n-1)}\frac{p(n-1)}{n-p}\rVert \nabla u \lVert_{L^{p}(\mathbf{R}^N)}.
\]

\paragraph{Step 4: $1<p <n$ and $u\in W^{1,p}$}
This is done as step 2.

This proof is due to Gagliardo and Nirenberg, who were the first to prove the inequality for $p=1$. This proof can be also found in \cite{HB,JJ,MW}.

\begin{thebibliography}{99}
\bibitem{HB} Haïm \textsc{Brezis}, \emph{Analyse fonctionnelle, Théorie et applications}, Mathématiques appliquées pour la maîtrise, Masson, Paris, 1983. \PMlinkexternal{[MR85a:46001]}{http://www.ams.org/mathscinet-getitem?mr=0697382}

\bibitem{JJ} Jürgen \textsc{Jost}, \emph{Partial Differential Equations}, Graduate Texts in Mathematics,  Springer, 2002,  \PMlinkexternal{[MR:2003f:35002]}{http://www.ams.org/mathscinet-getitem?mr=1919991}.

\bibitem{MW} Michel \textsc{Willem}, \emph{Analyse fonctinnelle élémentaire}, Cassini, Paris, 2003.
\end{thebibliography}
%%%%%
%%%%%
\end{document}
