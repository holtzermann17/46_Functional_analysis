\documentclass[12pt]{article}
\usepackage{pmmeta}
\pmcanonicalname{CompactnessOfClosedUnitBallInNormedSpaces}
\pmcreated{2013-03-22 17:48:41}
\pmmodified{2013-03-22 17:48:41}
\pmowner{asteroid}{17536}
\pmmodifier{asteroid}{17536}
\pmtitle{compactness of closed unit ball in normed spaces}
\pmrecord{13}{40274}
\pmprivacy{1}
\pmauthor{asteroid}{17536}
\pmtype{Theorem}
\pmcomment{trigger rebuild}
\pmclassification{msc}{46B50}
\pmsynonym{closed unit ball in a normed space is compact iff the space is finite dimensional}{CompactnessOfClosedUnitBallInNormedSpaces}

\endmetadata

% this is the default PlanetMath preamble.  as your knowledge
% of TeX increases, you will probably want to edit this, but
% it should be fine as is for beginners.

% almost certainly you want these
\usepackage{amssymb}
\usepackage{amsmath}
\usepackage{amsfonts}

% used for TeXing text within eps files
%\usepackage{psfrag}
% need this for including graphics (\includegraphics)
%\usepackage{graphicx}
% for neatly defining theorems and propositions
%\usepackage{amsthm}
% making logically defined graphics
%%%\usepackage{xypic}

% there are many more packages, add them here as you need them

% define commands here

\begin{document}
\PMlinkescapeword{Lemma}
\PMlinkescapephrase{finite dimensional}
\PMlinkescapephrase{infinite-dimensional}
\PMlinkescapephrase{subspace}
\PMlinkescapephrase{convergent}

{\bf Theorem -} Let $X$ be a normed space and $\overline{B_1(0)} \subset X$ the closed \PMlinkescapetext{unit} \PMlinkname{ball}{Ball}. Then $\overline{B_1(0)}$ is compact if and only if $X$ is \PMlinkname{finite dimensional}{Dimension2}.

The above result is false, in general, if one is considering other topologies in $X$ besides the norm topology (see, for example, the Banach-Alaoglu theorem). It follows that \PMlinkname{infinite dimensional}{Dimension2} normed spaces are not locally compact.

$\,$

{\bf \emph{Proof:}}
\begin{itemize}
\item $(\Longleftarrow)$ This is the easy part. Since $X$ is finite dimensional it is isomorphic to some $\mathbb{R}^n$ (with the standard topology). The result then follows from the Heine-Borel theorem.

$\,$

\item $(\Longrightarrow)$ Suppose that $X$ is not finite dimensional. Pick an element $x_1 \in X$ such that $\|x_1\|=1$ and denote by $S_1$ the \PMlinkname{subspace}{VectorSubspace} generated by $x_1.$

Recall that, according to the \PMlinkname{Riesz Lemma}{RiezsLemma}, there exists an element $x_2 \in X$ such that $\|x_2\|=1$ and $d(x_2,S_1)\geq \frac{1}{2}$, where $d(y, M):= \inf\{\|y-z\|: z\in M\}$ denotes the distance between an element $y\in X$ and a subspace $M\subset X.$

Now consider the subspace $S_2$ generated by $x_1, x_2$. Since $X$ is infinite dimensional $S_2$ is a proper subspace and we can still apply the Riesz Lemma to find an element $x_3$ such that $\|x_3\|=1$ and $d(x_3,S_2)\geq \frac{1}{2}.$

If we proceed inductively, we will find a sequence $\{x_n\}$ of norm 1 elements and a sequence of subspaces $S_n:=\langle x_1, \dots, x_n\rangle$ such that $d(x_{n+1}, S_n)\geq\frac{1}{2}.$ Under this setting it is easily seen that the sequence $\{x_n\}$ is in $\overline{B_1(0)}$ and satisfies $\|x_n - x_m\|\geq\frac{1}{2}$ for all $m \neq n \in \mathbb{N}$. Therefore, $\{x_n\}$ is a sequence in $\overline{B_1(0)}$ that has no \PMlinkname{convergent}{ConvergentSequence} subsequence, i.e., $\overline{B_1(0)}$ is not compact. $\square$
\end{itemize}

$\,$

{\bf Remarks on the proof -} Note that the sequence $\{x_n\}$ constructed in the proof does not have a \PMlinkname{Cauchy subsequence}{CauchySequence}. Thus we have in fact proven the slightly stronger result that $X$ is finite dimensional if and only if every bounded sequence in $X$ has a Cauchy subsequence. 

For Hilbert spaces the proof would be slightly simpler because one could just pick any orthonormal basis $\{e_n\},$ and it would \PMlinkescapetext{satisfy} $\|e_n - e_m\| = \sqrt{2}$ for all $m, n \in \mathbb{N}$ with $m\neq n,$ therefore having no convergent subsequence. For general normed spaces we cannot just pick orthonormal elements, since this notion does not exist. Thus, we have to use Riesz Lemma to assure the existence of elements with some \PMlinkescapetext{similar} \PMlinkescapetext{properties}.
%%%%%
%%%%%
\end{document}
