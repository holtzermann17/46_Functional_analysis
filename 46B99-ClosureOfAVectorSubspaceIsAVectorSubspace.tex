\documentclass[12pt]{article}
\usepackage{pmmeta}
\pmcanonicalname{ClosureOfAVectorSubspaceIsAVectorSubspace}
\pmcreated{2013-03-22 15:00:19}
\pmmodified{2013-03-22 15:00:19}
\pmowner{loner}{106}
\pmmodifier{loner}{106}
\pmtitle{closure of a vector subspace is a vector subspace}
\pmrecord{8}{36710}
\pmprivacy{1}
\pmauthor{loner}{106}
\pmtype{Theorem}
\pmcomment{trigger rebuild}
\pmclassification{msc}{46B99}
\pmclassification{msc}{15A03}
\pmclassification{msc}{54A05}
\pmrelated{ClosureOfAVectorSubspaceIsAVectorSubspace}
\pmrelated{ClosureOfSetsClosedUnderAFinitaryOperation}

\endmetadata

% this is the default PlanetMath preamble.  as your knowledge
% of TeX increases, you will probably want to edit this, but
% it should be fine as is for beginners.

% almost certainly you want these
\usepackage{amssymb}
\usepackage{amsmath}
\usepackage{amsfonts}
\usepackage{amsthm}

\usepackage{mathrsfs}

% used for TeXing text within eps files
%\usepackage{psfrag}
% need this for including graphics (\includegraphics)
%\usepackage{graphicx}
% for neatly defining theorems and propositions
%
% making logically defined graphics
%%%\usepackage{xypic}

% there are many more packages, add them here as you need them

% define commands here

\newcommand{\sR}[0]{\mathbb{R}}
\newcommand{\sC}[0]{\mathbb{C}}
\newcommand{\sN}[0]{\mathbb{N}}
\newcommand{\sZ}[0]{\mathbb{Z}}

 \usepackage{bbm}
 \newcommand{\Z}{\mathbbmss{Z}}
 \newcommand{\C}{\mathbbmss{C}}
 \newcommand{\R}{\mathbbmss{R}}
 \newcommand{\Q}{\mathbbmss{Q}}



\newcommand*{\norm}[1]{\lVert #1 \rVert}
\newcommand*{\abs}[1]{| #1 |}



\newtheorem{thm}{Theorem}
\newtheorem{defn}{Definition}
\newtheorem{prop}{Proposition}
\newtheorem{lemma}{Lemma}
\newtheorem{cor}{Corollary}
\begin{document}
\begin{thm} In a topological vector space 
the \PMlinkname{closure}{Closure} of a vector subspace is a vector subspace. 
\end{thm}

\begin{proof}
Let $X$ be the topological vector space over $\mathbbmss{F}$ where 
$\mathbbmss{F}$ is either $\R$ or $\C$, let $V$ be a vector subspace
in $X$, and let $\overline{V}$ be the closure of $V$. 
To prove that $\overline{V}$
is a vector subspace of $X$, it suffices
to prove that $\overline{V}$ is non-empty, and
$$
   \lambda x + \mu y \in \overline{V}
$$
whenever $\lambda,\mu \in \mathbbmss{F}$ and $x,y\in \overline{V}$. 

First, as $V\subseteq \overline{V}$, $\overline{V}$ contains the zero vector,
and $\overline{V}$ is non-empty. 
Suppose $\lambda,\mu,x,y$ are as above. 
Then there are nets $(x_i)_{i \in I}$, $(y_j)_{j \in J}$ in $V$ converging to 
$x,y$, respectively. 
In a topological vector space, addition and multiplication are continuous
operations. It follows that there is a net $(\lambda x_k + \mu y_k)_{k \in K}$ that converges to $\lambda x + \mu y$.

We have proven that $\lambda x + \mu y \in \overline{V}$, so 
$\overline{V}$ is a vector subspace.
\end{proof}
%%%%%
%%%%%
\end{document}
