\documentclass[12pt]{article}
\usepackage{pmmeta}
\pmcanonicalname{BoundedOperatorsOnAHilbertSpaceFormACalgebra}
\pmcreated{2013-03-22 14:47:12}
\pmmodified{2013-03-22 14:47:12}
\pmowner{HkBst}{6197}
\pmmodifier{HkBst}{6197}
\pmtitle{bounded operators on a Hilbert space form a $C^*$-algebra}
\pmrecord{9}{36438}
\pmprivacy{1}
\pmauthor{HkBst}{6197}
\pmtype{Result}
\pmcomment{trigger rebuild}
\pmclassification{msc}{46L05}
\pmrelated{RepresentationOfAC_cG_dTopologicalAlgebra}

\endmetadata

% this is the default PlanetMath preamble.  as your knowledge
% of TeX increases, you will probably want to edit this, but
% it should be fine as is for beginners.

% almost certainly you want these
\usepackage{amssymb}
\usepackage{amsmath}
\usepackage{amsfonts}

% used for TeXing text within eps files
%\usepackage{psfrag}
% need this for including graphics (\includegraphics)
%\usepackage{graphicx}
% for neatly defining theorems and propositions
%\usepackage{amsthm}
% making logically defined graphics
%%%\usepackage{xypic}

% there are many more packages, add them here as you need them

% define commands here
\newenvironment{lm}{\par\noindent\textbf{Lemma }}{}
\newenvironment{prf}{\par\noindent\textbf{Proof: }}{$\Box$}
\begin{document}
In this entry we show how the algebra $\mathrm{B}(H)$ of bounded linear operators on an Hilbert space $H$ is one of the most natural examples of \PMlinkname{$C^*$-algebras}{CAlgebra}. In fact, by the Gelfand-Naimark representation theorem, every $C^*$-algebra is isomorphic to a *-subalgebra of $\mathrm{B}(H)$ for some Hilbert space $H$.

\begin{lm}
If $H$ is a Hilbert space, then $\mathrm{B}(H)$, the algebra of bounded linear operators on $H$, is a $*$-algebra.
\end{lm}
\begin{prf}
Let $H$ be a Hilbert space. We must prove that the adjugation is an involution.
Let $\{P, Q\} \subset \mathrm{B}(H)$ and $l \in \mathbf{C}$. For every $\{x, y\} \subset H$ we have
\begin{enumerate}
\item $\langle P^{**} x | y \rangle = \langle x | P^*y \rangle = \langle P x | y \rangle$ so $P^{**} = P$,
\item $\langle (PQ)^* x | y \rangle = \langle x | PQ y \rangle = \langle P^* x | Q y \rangle = \langle Q^* P^* x | y \rangle$ so $(PQ)^* = Q^*\
P^*$ and
\item $\langle (lP+Q)^* x | y \rangle = \langle x | (lP+Q) y \rangle = l \langle x | Py \rangle + \langle x | Qy \rangle = l\langle P^* x | y \rangle + \langle Q^* x | y \rangle = \langle (l^*P^* + Q^*) x | y \rangle$ so $(lP+Q)^* = l^*P^* + Q^*$,
\end{enumerate}
so we see that the adjugation is an involution and thus $\mathrm{B}(H)$ is a $*$-algebra.
\end{prf}

\begin{lm}
If $H$ is a Hilbert space, then $\mathrm{B}(H)$ is a Banach algebra.
\end{lm}
\begin{prf}
Let $H$ be a Hilbert space and let $\{P, Q\} \subset \mathrm{B}(H)$. We have
$$\|PQ\| = \sup_{x \in H \setminus \{0\} } \frac{\| PQx \|_H}{\| x \|_H} \le \sup_{x \in H \setminus \{0\} } \frac{\|P\| \| Qx \|_H}{\| x \|_H\
} = \|P\| \|Q\|,$$
so we see that $\mathrm{B}(H)$ is a Banach algebra.
\end{prf}

\begin{lm}
If $H$ is a Hilbert space, then $\mathrm{B}(H)$ is a $C^*$-algebra.
\end{lm}
\begin{prf}
Let $H$ be a Hilbert space and let $P \in \mathrm{B}(H)$. We have
\begin{eqnarray*}
\|P\|^2 &=& \sup_{x \in H \setminus \{0\}} \frac{\|Px\|_H^2}{\|x\|_H^2} = \sup_{x \in H \setminus \{0\}} \frac{\langle Px | Px \rangle}{\|x\|_H^2} = \sup_{x \in H \setminus \{0\}} \frac{\langle P^*Px | x \rangle}{\|x\|_H^2}\\
&\le& \sup_{x \in H \setminus \{0\}} \frac{\|P^*Px\|_H \|x\|_H}{\|x\|_H^2} = \sup_{x \in H \setminus \{0\} } \frac{\| P^*Px \|_H}{\| x \|_H} = \|P^*P\|
\end{eqnarray*}
so $\|P\|^2 \le \|P^*P\|$ and because of the previous two lemmas say $\mathrm{B}(H)$ is a Banach algebra with involution it is a $C^*$-algebra.
\end{prf}

\begin{lm}
If $H$ is a Hilbert space, then every closed $*$-subalgebra of $\mathrm{B}(H)$ is a $C^*$-algebra.
\end{lm}
\begin{prf}
Let $A$ be a closed $*$-subalgebra of $\mathrm{B}(H)$. Because $A$ is a closed subspace of a Banach space it is itself a Banach space and thus a Banach algebra with an involution and also a $C^*$-algebra.
\end{prf}
%%%%%
%%%%%
\end{document}
