\documentclass[12pt]{article}
\usepackage{pmmeta}
\pmcanonicalname{FiniteRankApproximationOnSeparableHilbertSpaces}
\pmcreated{2013-03-22 18:23:18}
\pmmodified{2013-03-22 18:23:18}
\pmowner{karstenb}{16623}
\pmmodifier{karstenb}{16623}
\pmtitle{finite rank approximation on separable Hilbert spaces}
\pmrecord{10}{41033}
\pmprivacy{1}
\pmauthor{karstenb}{16623}
\pmtype{Theorem}
\pmcomment{trigger rebuild}
\pmclassification{msc}{46B99}

\usepackage{amssymb}
\usepackage{amsmath}
\usepackage{amsfonts}
\usepackage{amsthm}
\usepackage[sort&compress]{natbib}

%\usepackage{psfrag}
%\usepackage{graphicx}
%%%\usepackage{xypic}


% commands
%\newcommand{\comment}[1]{\small{(\,\textit{#1}\;)}}
\newcommand{\scal}[2]{\langle #1, #2 \rangle}
\begin{document}
\textbf{Theorem} Let $\mathcal{H}$ be a separable Hilbert space and let $T \in L(\mathcal{H})$. Then $T$ is a compact operator iff there is a sequence $\{F_n\}$ of finite rank operators with $\|T - F_n\| \to 0$.

\begin{proof}
$(\Rightarrow)$: Assume $T$ is compact on $\mathcal{H}$ and $\{e_n\}$ is an orthonormal basis of $\mathcal{H}$. Define:
\begin{align*}
P_n f &= \sum_{k=0}^n \scal{f}{e_k} e_k 
\end{align*}

It is clear that the $P_n$ have finite rank and that we have $\Vert P_n f\Vert \leq \Vert f\Vert$ for all $n \in \mathbb{N}$, $f \in \mathcal{H}$. 

Let $\mathcal{B}$ be the unit ball in $\mathcal{H}$. We have that $P_n \to I$ pointwise. Since the $P_n$ are contractive they are equicontinuous, hence $P_n$ converges uniformly to $I$ on compact sets, and in particular on $ \overline{T(\mathcal{B})}$, which is compact by assumption.
Therefore $P_n T \to T$ uniformly on $\mathcal{B}$, hence $\|P_n T - T\| \to 0$. 
Since $P_n T$ is bounded and of finite rank the first direction follows.

$(\Leftarrow)$: Now let $\{F_n\}$ be a sequence of bounded operators of finite rank with $\|T - F_n\| \to 0$. 
We have to show that $T(\mathcal{B})$ is relatively compact in $\mathcal{H}$. This is equivalent to $T(\mathcal{B})$ being totally bounded in $\mathcal{H}$.
So we are left to show that for all $\epsilon > 0$ there is an $\epsilon$-net $x_1, \cdots, x_n \in \mathcal{H}$ so that:
\begin{align*}
T(\mathcal{B}) &\subseteq \bigcup_{k=1}^n B_{\epsilon}(x_k)
\end{align*}

So choose $\epsilon > 0$ and $n \in \mathbb{N}$ fixed so that:
\begin{displaymath}
\|F_n - T\| < \frac{\epsilon}{2}
\end{displaymath}

Choose $x_1, \cdots, x_m \in \mathcal{H}$ with:
\begin{align*}
F_n(\mathcal{B}) &\subseteq \bigcup_{k=1}^m B_{\frac{\epsilon}{2}}(x_k)
\end{align*}

Hence (by the triangle inequality):
\begin{align*}
T(\mathcal{B}) &\subseteq \bigcup_{k=1}^m B_{\epsilon}(x_k)
\end{align*}

and we are done. 
\end{proof}
%%%%%
%%%%%
\end{document}
